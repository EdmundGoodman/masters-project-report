\chapter{Related work}
\label{chap:related-work}

% This chapter covers relevant (and typically, recent) research
% which you build upon (or improve upon). There are two complementary
% goals for this chapter:
% \begin{enumerate}
%   \item to show that you know and understand the state of the art; and
%   \item to put your work in context
% \end{enumerate}
%
% Ideally you can tackle both together by providing a critique of
% related work, and describing what is insufficient (and how you do
% better!)
%
% The related work chapter should usually come either near the front or
% near the back of the dissertation. The advantage of the former is that
% you get to build the argument for why your work is important before
% presenting your solution(s) in later chapters; the advantage of the
% latter is that don't have to forward reference to your solution too
% much. The correct choice will depend on what you're writing up, and
% your own personal preference.




\section{Static and dynamic languages}
\label{sec:static-dynamic-languages}

% Hook
% Argument
% Link

% Precisely define dynamism


% An example of where C++ dynamism presents an optimisation boundary that would otherwise be found by the compiler (and possibly that a JIT can find it?)

% Object orientated optimisations (virtual dispatch in C++/objective-C)
% Why is JS faster than Python - more constrained






\section{JIT Compilation}
\label{sec:jit-compilation}

\subsection{Copy-and-patch compilation}
\label{ssec:copy-and-patch-compilation}

\subsection{Lua JIT}
\label{ssec:lua-jit}

\subsection{GraalVM}
\label{ssec:graalvm}

\subsection{PyPI}
\label{ssec:graalvm}

\subsection{Numba, JAX, and PyTorch}
\label{ssec:number-jax-pytorch}



% \section{Python performance}
% \label{sec:python-performance}

% % \subsection{Python performance matters}
% % \label{ssec:python-performance-matters}

% Typically, we want to bind into C as much as possible -- see things like Scalene to measure this. However, for our pattern rewriting workload, this is less suitable

% \subsection{Python performance measurement}
% \label{ssec:python-perf-measurement}
% timeit/asv/pyperf and stuff maybe? Possibly not worth discussing


\subsection{Faster CPython}
\label{ssec:faster-cpython}

% Timeline

% Hook
Python is a notoriously slow language \cite{}. The Faster CPython project is an attempt to remedy this fact.
% Argument
Over the course of the recent CPython major versions, new optimisations have been incrementally added as part of this project, resulting in incremental performance gains (\autoref{tab:faster-cpython}).
% Link
This section discusses these optimisations, and their effect on CPython's performance.

% Hook
% Argument
% Link

\begin{table}[H]
  \caption{Incremental performance gains on the PyPerformance benchmark suite achieved by optimisations to the CPython interpreter.}
  \label{tab:faster-cpython}
  \centering
  \begin{tabular}{lll}
    \toprule
    \textbf{Python version} & \textbf{Optimisation over previous} & \textbf{PyPerformance result} \\
    \midrule
    CPython 3.10.17 & Baseline & $x$ \\
    CPython 3.11.12 & Specialising adaptive interpreter & $x$ \\
    CPython 3.12.10 & Comprehension inlining & $x$ \\
    CPython 3.13.3 & Version bump & $x$ \\
    CPython 3.13.3 & Enabled experimental JIT & $x$ \\
    CPython 3.14.0a7 & Version bump & $x$ \\
    CPython 3.14.0a7 & Enabled tail call interpreter & $x$ \\
    % \midrule
    % PyPy 3.11.11 & JIT compilation & $x$ \\
    \bottomrule
  \end{tabular}
\end{table}

\subsubsection{Specialising adaptive interpreter}
\label{sssec:graalvm}

\subsubsection{Comprehension inlining}
\label{sssec:graalvm}

\subsubsection{Experimental JIT compiler}
\label{sssec:graalvm}

\subsubsection{Tail call interpreter}
\label{sssec:graalvm}

