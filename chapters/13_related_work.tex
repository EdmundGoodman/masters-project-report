\chapter{Related work}
\label{chap:related-work}

% This chapter covers relevant (and typically, recent) research
% which you build upon (or improve upon). There are two complementary
% goals for this chapter:
% \begin{enumerate}
%   \item to show that you know and understand the state of the art; and
%   \item to put your work in context
% \end{enumerate}
%
% Ideally you can tackle both together by providing a critique of
% related work, and describing what is insufficient (and how you do
% better!)
%
% The related work chapter should usually come either near the front or
% near the back of the dissertation. The advantage of the former is that
% you get to build the argument for why your work is important before
% presenting your solution(s) in later chapters; the advantage of the
% latter is that don't have to forward reference to your solution too
% much. The correct choice will depend on what you're writing up, and
% your own personal preference.

\section{Python performance}
\label{sec:python-performance}

% \subsection{Python performance matters}
% \label{ssec:python-performance-matters}

PyPI

Numba

\subsection{Pyfaster}
\label{ssec:pyfaster}

\subsection{Copy-and-patch compilation}
\label{ssec:copy-and-patch-compilation}

\section{Understanding the performance of compiler frameworks}
\label{sec:understanding-framework-performance}

\subsection{How slow is MLIR?}
\label{ssec:how-slow-is-mlir}

\section{Code rewriting and JIT compilation}
\label{sec:code-rewriting-jit}

\subsection{Lua JIT}
\label{ssec:lua-jit}

\subsection{GraalVM}
\label{ssec:graalvm}