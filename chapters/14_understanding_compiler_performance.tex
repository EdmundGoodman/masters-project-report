% % \chapter{Design and implementation}

% % This chapter may be called something else\ldots but in general the
% % idea is that you have one (or a few) ``meat'' chapters which describe
% % the work you did in technical detail.

% % \chapter{Measuring Python's performance}
% % \label{chap:understanding-compiler-performance}


% \chapter{Understanding compiler performance}
% \label{chap:understanding-compiler-performance}

% % \textit{
% % \textbf{TODO:}
% % \begin{enumerate}
% %     \item What performance do we care about (compilation time rather than performance of generated binary)?
% %     \item Why do we care about (compilation time) performance?
% %     \item What question(s) are we trying to answer?
% %     \begin{enumerate}
% %         \item How does xDSL compare to MLIR, particularly in relation to dynamism of workloads/language runtimes?
% %         \item What are bottlenecks in xDSL which we could mitigate in the next chapter?
% %     \end{enumerate}
% %     \item How did we go about answering them?
% % \end{enumerate}
% % }





% \subsection{Operation instantiation}
% \label{ssec:op-instantiation}

% % % What is the overall trend here?
% % % Are we measuring the implementation of the language?
% % % What does this say about dynamism?
% % % What are opportunities for optimisation?


% % Operations are a fundamental data structure in MLIR-like compilers. As such, their instantiation is a frequent task. Both xDSL and MLIR offer various functions to do this. The \texttt{create} functions directly construct the operation, builders offer a sugar over this construction, and cloning copies all data associated with an existing operation.

% % \subsubsection{Creating operations}
% % \label{sssec:ubenchmark-op-instantiation-creating}

% % \subsubsection{Building operations}
% % \label{sssec:ubenchmark-op-instantiation-building}

% % \subsubsection{Cloning operations}
% % \label{sssec:ubenchmark-op-instantiation-cloning}

% % \subsubsection{Discussion}
% % \label{sssec:ubenchmark-op-instantiation-discussion}

% % \subsection{All microbenchmarks}
% % \label{ssec:ubenchmark-all}




% \section{End-to-end benchmarking}
% \label{sec:e2e-benchmark}


% \section{Benchmarking xDSL's pattern rewriter}
% \label{sec:rewriter-benchmark}


% \section{Quantifying dynamism in MLIR and xDSL}
% \label{sec:quantifying-dynamism}