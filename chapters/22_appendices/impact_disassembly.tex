\chapter{Disassembly of dynamic dispatch experiments}
\label{chap:impact-disassembly}


\vspace{2em}
\begin{code}
    \begin{minted}[fontsize=\scriptsize]{text}
main:
 stp	x29, x30, [sp, #-32]!
 str	x19, [sp, #16]
 mov	x29, sp
 ldr	x0, [x1, #8]
 mov	x19, x1
 mov	x1, xzr
 mov	w2, #0xa                   	// #10
 bl	0 <strtol>
    R_AARCH64_CALL26 strtol
 ldr	x8, [x19, #16]
 mov	x19, x0
 mov	x1, xzr
 mov	w2, #0xa                   	// #10
 mov	x0, x8
 bl	0 <strtol>
    R_AARCH64_CALL26 strtol
// ===== Start of function invocation ===== //
 sub	w0, w19, w0
// ====== End of function invocation ====== //
 ldr	x19, [sp, #16]
 ldp	x29, x30, [sp], #32
 ret
    \end{minted}
    \caption{Disassembly of inlined function invocation (Listing \ref{listing:impact-dispatch-definition} \circledbase{pairedTwoDarkBlue}{\scriptsize{1}}).}
    \label{listing:impact-dispatch-inlined-disassembly}
\end{code}


\vspace{2em}
\begin{code}
    \begin{minted}[fontsize=\scriptsize]{text}
main:
 stp	x29, x30, [sp, #-32]!
 str	x19, [sp, #16]
 mov	x29, sp
 ldr	x0, [x1, #8]
 mov	x19, x1
 mov	x1, xzr
 mov	w2, #0xa                   	// #10
 bl	0 <strtol>
    R_AARCH64_CALL26 strtol
 ldr	x8, [x19, #16]
 mov	x19, x0
 mov	x1, xzr
 mov	w2, #0xa                   	// #10
 mov	x0, x8
 bl	0 <strtol>
    R_AARCH64_CALL26 strtol
// ===== Start of function invocation ===== //
 mov	x2, x0
 add	x0, x29, #0x1f
 mov	w1, w19
 bl	0 <main>
    R_AARCH64_CALL26 Base::uninlinedFunc(int, int)
// ====== End of function invocation ====== //
 ldr	x19, [sp, #16]
 ldp	x29, x30, [sp], #32
 ret
Base::uninlinedFunc(int, int):
 sub	w0, w1, w2
 ret
    \end{minted}
    \caption{Disassembly of uninlined function invocation (Listing \ref{listing:impact-dispatch-definition} \circledbase{pairedThreeLightGreen}{\scriptsize{2}}).}
    \label{listing:impact-dispatch-uninlined-disassembly}
\end{code}


\vspace{2em}
\begin{code}
    \begin{minted}[fontsize=\scriptsize]{text}
main:
 sub	sp, sp, #0x30
 stp	x29, x30, [sp, #16]
 stp	x20, x19, [sp, #32]
 add	x29, sp, #0x10
 ldr	x0, [x1, #8]
 mov	x19, x1
 mov	x1, xzr
 mov	w2, #0xa                   	// #10
 bl	0 <strtol>
    R_AARCH64_CALL26 strtol
 ldr	x8, [x19, #16]
 mov	x20, x0
 mov	x1, xzr
 mov	w2, #0xa                   	// #10
 mov	x0, x8
 bl	0 <strtol>
    R_AARCH64_CALL26 strtol
 ldr	x8, [x19, #24]
 mov	x19, x0
 mov	x1, xzr
 mov	w2, #0xa                   	// #10
 mov	x0, x8
 bl	0 <strtol>
    R_AARCH64_CALL26 strtol
// ===== Start of function invocation ===== //
 adrp	x8, 0 <main>
    R_AARCH64_ADR_PREL_PG_HI21 vtable for Base+0x10
 add	x11, x8, #0x0
    R_AARCH64_ADD_ABS_LO12_NC vtable for Base+0x10
 cmp	w0, #0x0
 adrp	x8, 0 <main>
    R_AARCH64_ADR_PREL_PG_HI21 vtable for Derived+0x10
 add	x8, x8, #0x0
    R_AARCH64_ADD_ABS_LO12_NC vtable for Derived+0x10
 mov	x9, sp
 add	x10, sp, #0x8
 stp	x8, x11, [sp]
 mov	w1, w20
 csel	x0, x10, x9, gt
 mov	w2, w19
 ldr	x8, [x0]
 ldr	x8, [x8]
 blr	x8
// ====== End of function invocation ====== //
 ldp	x20, x19, [sp, #32]
 ldp	x29, x30, [sp, #16]
 add	sp, sp, #0x30
 ret
Base::virtualFunc(int, int):
 sub	w0, w1, w2
 ret
Derived::virtualFunc(int, int):
 sub	w0, w2, w1
 ret
    \end{minted}
    \caption{Disassembly of polymorphic function invocation (Listing \ref{listing:impact-dispatch-definition} \circledbase{pairedFourDarkGreen}{\scriptsize{3}}).}
    \label{listing:impact-dispatch-polymorphic-disassembly}
\end{code}


\vspace{2em}
\begin{code}
    \begin{minted}[fontsize=\scriptsize]{text}
//// Trace of `Extensibility.time_invoke_method_baseline` :
// == microbenchmarks:152 `time_invoke_method_baseline` ==
// >>> a = 5
154         0   LOAD_CONST           1   (5)
            2   STORE_FAST           1   (a)
// >>> b = 6
155         4   LOAD_CONST           2   (6)
            6   STORE_FAST           2   (b)
// >>> _ = None  # Simulate passing arguments
156         8   LOAD_CONST           3   (None)
            10  STORE_FAST           3   (_)
// >>> Extensibility.EXAMPLE.regularFunction
157         12  LOAD_GLOBAL          0   (Extensibility)
            14  LOAD_ATTR            1   (EXAMPLE)
            16  LOAD_ATTR            2   (regularFunction)
            18  POP_TOP                  ()
// >>> return a - b
158         20  LOAD_FAST            1   (a)
            22  LOAD_FAST            2   (b)
            24  BINARY_SUBTRACT          ()
            26  RETURN_VALUE             ()
// =============================================

// == microbenchmarks:160 `time_invoke_method` ==
// >>> a = 5
162         0   LOAD_CONST           1   (5)
            2   STORE_FAST           1   (a)
// >>> b = 6
163         4   LOAD_CONST           2   (6)
            6   STORE_FAST           2   (b)
// >>> return Extensibility.EXAMPLE.regularFunction(a, b)
164         8   LOAD_GLOBAL          0   (Extensibility)
            10  LOAD_ATTR            1   (EXAMPLE)
            12  LOAD_METHOD          2   (regularFunction)
            14  LOAD_FAST            1   (a)
            16  LOAD_FAST            2   (b)
            18  CALL_METHOD          2   ()

    // === microbenchmarks:16 `regularFunction` ====
    // >>> return a - b
     17         0   LOAD_FAST            1   (a)
                2   LOAD_FAST            2   (b)
                4   BINARY_SUBTRACT          ()
                6   RETURN_VALUE             ()
    // =============================================
            20  RETURN_VALUE             ()
// =============================================
    \end{minted}
    \caption{Bytecode trace of Python method invocation, with the \texttt{CALL\_METHOD} and \texttt{RETURN\_VALUE} opcodes being the target of the measurement by subtracting the elapsed time from that of the baseline inlined implementation.}
    \label{listing:impact-dispatch-python-disassembly}
\end{code}
