\chapter{MLIR workloads}
\label{chap:mlir-workloads}

\section{Constant folding}
\label{sec:constant-folding-workload-impl}

% Hook
% Argument
% Link

\vspace{2em}

\begin{code}
    \begin{minted}[fontsize=\footnotesize]{text}
def constant_folding_module(size: int) -> ModuleOp:
    """Generate a constant folding workload of a given size.

    The output of running the command
    `print(WorkloadBuilder().constant_folding_module(size=5))` is shown
    below:

    ```mlir
    "builtin.module"() ({
        %0 = "arith.constant"() {"value" = 865 : i32} : () -> i32
        %1 = "arith.constant"() {"value" = 395 : i32} : () -> i32
        %2 = "arith.addi"(%1, %0) : (i32, i32) -> i32
        %3 = "arith.constant"() {"value" = 777 : i32} : () -> i32
        %4 = "arith.addi"(%3, %2) : (i32, i32) -> i32
        %5 = "arith.constant"() {"value" = 912 : i32} : () -> i32
        "test.op"(%4) : (i32) -> ()
    }) : () -> ()
    ```
    """
    assert size >= 0
    random.seed(RANDOM_SEED)
    ops: list[Operation] = []
    ops.append(ConstantOp(IntegerAttr(random.randint(1, 1000), i32)))
    for i in range(1, size + 1):
        if i % 2 == 0:
            ops.append(AddiOp(ops[i - 1], ops[i - 2]))
        else:
            ops.append(ConstantOp(IntegerAttr(random.randint(1, 1000), i32)))
    ops.append(TestOp([ops[(size // 2) * 2]]))
    return ModuleOp(ops)
    \end{minted}
    \caption{.}
    \label{listing:constant-folding-workload}
\end{code}










% \subsubsection{Loop unrolling}
% \label{sssec:experimental-workload-loop-unrolling}

% \begin{figure}[H]
%     \centering
%     \begin{subfigure}[b]{0.45\textwidth}
%        \centering
%         \begin{minted}[fontsize=\scriptsize]{text}
%             builtin.module {
%               %0 = arith.constant 0 : index
%               %1 = arith.constant 32 : index
%               %2 = arith.constant 1 : index
%               %3 = scf.for %4 = %0 to %1 step %2 iter_args(%5 = %0) -> (index) {
%                 %6 = arith.addi %4, %5 : index
%                 scf.yield %6 : index
%               }
%               "test.op"(%3) : (index) -> ()
%             }
%         \end{minted}
%         \footnotesize\vspace{2em}
%         \caption{Original \ac{ir}.}
%         \label{listing:loop-unrolling-workload-initial}
%     \end{subfigure}
%     \hfill
%     \begin{subfigure}[b]{0.45\textwidth}
%         \centering
%         \begin{minted}[breakanywhere,fontsize=\scriptsize]{text}
%             builtin.module {
%               %0 = arith.constant 0 : index
%               %1 = arith.constant 32 : index
%               %2 = arith.constant 4 : index
%               %3 = scf.for %4 = %0 to %1 step %2 iter_args(%5 = %0) -> (index) {
%                 %6 = arith.addi %4, %5 : index
%                 %7 = arith.addi %4, %6 : index
%                 %8 = arith.addi %4, %7 : index
%                 %9 = arith.addi %4, %8 : index
%                 scf.yield %9 : index
%               }
%               "test.op"(%3) : (index) -> ()
%             }
%         \end{minted}
%         \caption{\ac{ir} unrolled by a factor of four.}
%         \label{listing:loop-unrolling-workload-unrolled}
%     \end{subfigure}
%     \vspace{1em}
%     \captionsetup{name=Listing}
%     \caption{Example \ac{mlir} constant folding workload.}
%     \label{listing:loop-unrolling-workload-example}
% \end{figure}

% \vspace{2em}

%% TODO: If any more hero workloads like CIRCT or PDL are added, put them here!
