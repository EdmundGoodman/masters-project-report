\chapter*{Abstract}
% Write a summary of the whole thing. Make sure it fits on one page.


\makeatletter
{\large\textsl{\@title}}
\makeatother
\vspace{0.5em}

% //  1. Introduction. In one sentence, what's the topic?
MLIR is a modular compiler framework that provides core infrastructure to be leveraged and extended by users implementing their own compilers, an inherently dynamic design as a result of the underlying heterogeneous data structure whose shape is known only at runtime. %that bypasses its implementation language's type system as it relies on verification of invariants for user-provided abstractions.
% //  2. State the problem you tackle.
This approach presents an inherent optimization boundary, as the dynamic structures cannot be precisely reasoned about before runtime to guarantee the validity of optimisations, meaning static ahead-of-time compilation provides fewer benefits.
% This user-extensible approach presents an inherent optimization boundary, as the data structures and transformations provided by users cannot be specialized ahead of time, meaning static ahead-of-time compilation provides fewer benefits.
% //  3. Summarize (in one sentence) why nobody else has adequately answered the research question yet.
Previous compiler frameworks accept the limitations of this optimization boundary, leveraging only the remaining optimizations offered by static ahead-of-time compilation, yet still incurring the costs of long build times and reduced flexibility, suggesting dynamic languages might be more suitable.
% //  4. Explain, in one sentence, how you tackled the research question.
We examine performance bottlenecks incurred by dynamic languages for code rewriting tasks in xDSL, a Python-native compiler framework inspired by MLIR.
% //  5. In one sentence, how did you go about doing the research that follows from your big idea.
We find that both the inherent dynamism of these rewriting tasks over runtime heterogeneous data structures and modern interpreter optimisations narrow the performance gap between static and dynamic languages, using both traditional measurement techniques and a novel tool for performance profiling bytecode instructions.
% //  6. As a single sentence, what's the key impact of your research?
% Our research challenges the status quo of implementing user-extensible compiler frameworks in static, ahead-of-time compiled languages, instead motivating the use of dynamic languages which balance performance with flexibility and fast build times.
Our research challenges the status quo of implementing user-extensible compiler frameworks in static, ahead-of-time compiled languages.
% //  (7.) From Tobias' review, add a concluding sentence
Instead, we motivate the use of dynamic languages, demonstrating that they balance compilation performance with the flexibility and fast build times.
% Instead, we motivate the use of dynamic languages for these applications, demonstrating that they balance compilation performance with the flexibility and fast build times.

% \textbf{Keywords: } xDSL, MLIR, LLVM, Dynamic Programming Languages, Performance, User-extensible Compiler Infrastructure
