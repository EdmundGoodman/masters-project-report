\appendix

% \chapter{Technical details, proofs, etc.}

% Appendices are for optional materials that is not essential to
% understanding the work, and that the examiners are not expected to
% read, but that will be of value to readers interested in additional,
% in-depth technical detail.


\chapter{PyPerformance version comparison}
\label{chap:pyperformance-version-comparison}

% Hook
The following section describes the procedure and provides the raw results used to evaluate the speed-ups between CPython versions.
% Argument
Each of the CPython versions was compiled from source with the \texttt{--enable-optimizations} and \texttt{--with-lto} configuration flags. This compilation was performed on the experimental machine (detailed in \autoref{ssec:experimental-setup}) to avoid issues with cross-compilation.
The PyPerformance tool runs a suite of benchmarks, each resulting in their own speed-up value (tabulated for each version comparison in Tables \ref{listing:pyperformance-results-310-311}, \ref{listing:pyperformance-results-311-313}, and \ref{listing:pyperformance-results-313-313-jit}). These speed-up values are aggregated into a single speed-up value as the geometric mean. %, using a bash command (Listing \ref{listing:bash-geomean-pyperformance}).
A geometric mean is used because it is less skewed by outlying data. % TODO: Weights better than arithmetic for this use case + source
% Link
The calculated speed-ups are shown in the main body of the thesis (\autoref{tab:faster-cpython}).

\vspace{2em}

% \begin{code}
%     \begin{minted}[fontsize=\footnotesize]{text}
%         pyperformance run --python=BASELINE_PYTHON -f -o baseline_file.json
%         pyperformance run --python=CHANGED_PYTHON -f -o changed_file.json
%         pyperformance compare baseline_file.json changed_file.json -O table | \
%             grep -Eo "([0-9]+\.[0-9]+)x faster" | \
%             grep -Eo "([0-9]+\.[0-9]+)x faster" | \
%             grep -Eo "[0-9]+\.[0-9]+" | \
%             awk 'BEGIN { sum = 0; count = 0 } { sum += log($1); count++ } END { print exp(sum/count) }'
%     \end{minted}
%     \caption{Bash commands to calculate the geometric mean speedup across the benchmarks recorded by the PyPerformance tool.}
%     \label{listing:bash-geomean-pyperformance}
% \end{code}

\vspace{2em}

\begin{code}
    \begin{minted}[fontsize=\scriptsize]{text}
py310.json
==========

Performance version: 1.11.0
Report on Linux-6.8.0-1029-aws-x86_64-with-glibc2.39
Number of logical CPUs: 16
Start date: 2025-06-02 23:12:50.096047
End date: 2025-06-02 23:39:13.222073

py311.json
==========

Performance version: 1.11.0
Report on Linux-6.8.0-1029-aws-x86_64-with-glibc2.39
Number of logical CPUs: 16
Start date: 2025-06-02 22:42:35.309501
End date: 2025-06-02 23:09:34.936710

+-------------------------------+------------+------------+--------------+------------------------+
| Benchmark                     | py310.json | py311.json | Change       | Significance           |
+===============================+============+============+==============+========================+
| async_generators              | 496 ms     | 405 ms     | 1.23x faster | Significant (t=102.94) |
+-------------------------------+------------+------------+--------------+------------------------+
| async_tree_cpu_io_mixed       | 1.13 sec   | 959 ms     | 1.18x faster | Significant (t=24.21)  |
+-------------------------------+------------+------------+--------------+------------------------+
| async_tree_eager              | 854 ms     | 607 ms     | 1.41x faster | Significant (t=28.47)  |
+-------------------------------+------------+------------+--------------+------------------------+
| async_tree_eager_cpu_io_mixed | 1.14 sec   | 959 ms     | 1.18x faster | Significant (t=25.07)  |
+-------------------------------+------------+------------+--------------+------------------------+
| async_tree_eager_io           | 1.98 sec   | 1.39 sec   | 1.42x faster | Significant (t=115.69) |
+-------------------------------+------------+------------+--------------+------------------------+
| async_tree_eager_memoization  | 989 ms     | 753 ms     | 1.31x faster | Significant (t=70.85)  |
+-------------------------------+------------+------------+--------------+------------------------+
| async_tree_io                 | 1.98 sec   | 1.39 sec   | 1.42x faster | Significant (t=126.72) |
+-------------------------------+------------+------------+--------------+------------------------+
| async_tree_memoization        | 989 ms     | 754 ms     | 1.31x faster | Significant (t=80.45)  |
+-------------------------------+------------+------------+--------------+------------------------+
| async_tree_none               | 848 ms     | 607 ms     | 1.40x faster | Significant (t=28.13)  |
+-------------------------------+------------+------------+--------------+------------------------+
| asyncio_tcp                   | 907 ms     | 775 ms     | 1.17x faster | Significant (t=6.00)   |
+-------------------------------+------------+------------+--------------+------------------------+
| asyncio_tcp_ssl               | 2.32 sec   | 3.63 sec   | 1.56x slower | Significant (t=-11.69) |
+-------------------------------+------------+------------+--------------+------------------------+
| asyncio_websockets            | 643 ms     | 633 ms     | 1.02x faster | Not significant        |
+-------------------------------+------------+------------+--------------+------------------------+
| bench_mp_pool                 | 14.4 ms    | 12.8 ms    | 1.12x faster | Significant (t=7.41)   |
+-------------------------------+------------+------------+--------------+------------------------+
| bench_thread_pool             | 1.78 ms    | 1.69 ms    | 1.06x faster | Significant (t=31.34)  |
+-------------------------------+------------+------------+--------------+------------------------+
| chameleon                     | 9.92 ms    | 7.88 ms    | 1.26x faster | Significant (t=53.29)  |
+-------------------------------+------------+------------+--------------+------------------------+
| chaos                         | 130 ms     | 81.1 ms    | 1.61x faster | Significant (t=95.28)  |
+-------------------------------+------------+------------+--------------+------------------------+
| comprehensions                | 29.5 us    | 25.5 us    | 1.16x faster | Significant (t=90.45)  |
+-------------------------------+------------+------------+--------------+------------------------+
| coroutines                    | 35.7 ms    | 28.8 ms    | 1.24x faster | Significant (t=119.60) |
+-------------------------------+------------+------------+--------------+------------------------+
| coverage                      | 89.0 ms    | 84.8 ms    | 1.05x faster | Significant (t=5.95)   |
+-------------------------------+------------+------------+--------------+------------------------+
| create_gc_cycles              | 1.28 ms    | 1.11 ms    | 1.15x faster | Significant (t=24.86)  |
+-------------------------------+------------+------------+--------------+------------------------+
| crypto_pyaes                  | 134 ms     | 85.5 ms    | 1.56x faster | Significant (t=202.03) |
+-------------------------------+------------+------------+--------------+------------------------+
| dask                          | 525 ms     | 454 ms     | 1.16x faster | Significant (t=21.98)  |
+-------------------------------+------------+------------+--------------+------------------------+
| deepcopy                      | 503 us     | 407 us     | 1.24x faster | Significant (t=59.44)  |
+-------------------------------+------------+------------+--------------+------------------------+
| deepcopy_memo                 | 60.3 us    | 45.5 us    | 1.32x faster | Significant (t=77.90)  |
+-------------------------------+------------+------------+--------------+------------------------+
| deepcopy_reduce               | 4.45 us    | 3.59 us    | 1.24x faster | Significant (t=78.19)  |
+-------------------------------+------------+------------+--------------+------------------------+
| deltablue                     | 8.47 ms    | 4.08 ms    | 2.08x faster | Significant (t=150.55) |
+-------------------------------+------------+------------+--------------+------------------------+
| django_template               | 49.9 ms    | 38.9 ms    | 1.28x faster | Significant (t=45.49)  |
+-------------------------------+------------+------------+--------------+------------------------+
| docutils                      | 3.32 sec   | 2.70 sec   | 1.23x faster | Significant (t=110.18) |
+-------------------------------+------------+------------+--------------+------------------------+
| dulwich_log                   | 95.2 ms    | 79.0 ms    | 1.20x faster | Significant (t=59.47)  |
+-------------------------------+------------+------------+--------------+------------------------+
| fannkuch                      | 542 ms     | 434 ms     | 1.25x faster | Significant (t=33.10)  |
+-------------------------------+------------+------------+--------------+------------------------+
| float                         | 126 ms     | 86.2 ms    | 1.47x faster | Significant (t=127.64) |
+-------------------------------+------------+------------+--------------+------------------------+
| gc_traversal                  | 3.60 ms    | 3.56 ms    | 1.01x faster | Not significant        |
+-------------------------------+------------+------------+--------------+------------------------+
| generators                    | 60.4 ms    | 55.7 ms    | 1.08x faster | Significant (t=29.61)  |
+-------------------------------+------------+------------+--------------+------------------------+
| genshi_text                   | 33.8 ms    | 25.6 ms    | 1.32x faster | Significant (t=85.96)  |
+-------------------------------+------------+------------+--------------+------------------------+
| genshi_xml                    | 70.6 ms    | 61.1 ms    | 1.16x faster | Significant (t=29.82)  |
+-------------------------------+------------+------------+--------------+------------------------+
| go                            | 259 ms     | 166 ms     | 1.56x faster | Significant (t=120.34) |
+-------------------------------+------------+------------+--------------+------------------------+
| hexiom                        | 10.5 ms    | 7.43 ms    | 1.41x faster | Significant (t=34.56)  |
+-------------------------------+------------+------------+--------------+------------------------+
| html5lib                      | 106 ms     | 81.4 ms    | 1.30x faster | Significant (t=21.53)  |
+-------------------------------+------------+------------+--------------+------------------------+
| json_dumps                    | 14.7 ms    | 13.6 ms    | 1.08x faster | Significant (t=21.42)  |
+-------------------------------+------------+------------+--------------+------------------------+
| json_loads                    | 30.2 us    | 28.4 us    | 1.06x faster | Significant (t=23.39)  |
+-------------------------------+------------+------------+--------------+------------------------+
| logging_format                | 11.0 us    | 7.91 us    | 1.39x faster | Significant (t=50.52)  |
+-------------------------------+------------+------------+--------------+------------------------+
| logging_silent                | 197 ns     | 123 ns     | 1.60x faster | Significant (t=64.16)  |
+-------------------------------+------------+------------+--------------+------------------------+
| logging_simple                | 9.95 us    | 7.10 us    | 1.40x faster | Significant (t=39.79)  |
+-------------------------------+------------+------------+--------------+------------------------+
| mako                          | 17.3 ms    | 11.7 ms    | 1.47x faster | Significant (t=116.59) |
+-------------------------------+------------+------------+--------------+------------------------+
| mdp                           | 3.51 sec   | 3.20 sec   | 1.10x faster | Significant (t=16.77)  |
+-------------------------------+------------+------------+--------------+------------------------+
| meteor_contest                | 120 ms     | 109 ms     | 1.11x faster | Significant (t=58.59)  |
+-------------------------------+------------+------------+--------------+------------------------+
| nbody                         | 150 ms     | 102 ms     | 1.47x faster | Significant (t=21.91)  |
+-------------------------------+------------+------------+--------------+------------------------+
| nqueens                       | 111 ms     | 94.7 ms    | 1.17x faster | Significant (t=28.95)  |
+-------------------------------+------------+------------+--------------+------------------------+
| pathlib                       | 29.5 ms    | 27.6 ms    | 1.07x faster | Significant (t=14.71)  |
+-------------------------------+------------+------------+--------------+------------------------+
| pickle                        | 11.8 us    | 11.8 us    | 1.00x faster | Not significant        |
+-------------------------------+------------+------------+--------------+------------------------+
| pickle_dict                   | 28.5 us    | 29.5 us    | 1.03x slower | Significant (t=-8.47)  |
+-------------------------------+------------+------------+--------------+------------------------+
| pickle_list                   | 4.17 us    | 4.16 us    | 1.00x faster | Not significant        |
+-------------------------------+------------+------------+--------------+------------------------+
| pickle_pure_python            | 507 us     | 364 us     | 1.39x faster | Significant (t=201.27) |
+-------------------------------+------------+------------+--------------+------------------------+
| pidigits                      | 215 ms     | 208 ms     | 1.04x faster | Significant (t=16.13)  |
+-------------------------------+------------+------------+--------------+------------------------+
| pprint_pformat                | 2.30 sec   | 1.70 sec   | 1.35x faster | Significant (t=44.81)  |
+-------------------------------+------------+------------+--------------+------------------------+
| pprint_safe_repr              | 1.12 sec   | 825 ms     | 1.35x faster | Significant (t=62.68)  |
+-------------------------------+------------+------------+--------------+------------------------+
| pyflate                       | 747 ms     | 463 ms     | 1.61x faster | Significant (t=105.36) |
+-------------------------------+------------+------------+--------------+------------------------+
| python_startup                | 11.0 ms    | 10.2 ms    | 1.08x faster | Significant (t=113.04) |
+-------------------------------+------------+------------+--------------+------------------------+
| python_startup_no_site        | 7.18 ms    | 7.62 ms    | 1.06x slower | Significant (t=-63.11) |
+-------------------------------+------------+------------+--------------+------------------------+
| raytrace                      | 543 ms     | 355 ms     | 1.53x faster | Significant (t=97.20)  |
+-------------------------------+------------+------------+--------------+------------------------+
| regex_compile                 | 204 ms     | 161 ms     | 1.27x faster | Significant (t=68.71)  |
+-------------------------------+------------+------------+--------------+------------------------+
| regex_dna                     | 208 ms     | 182 ms     | 1.14x faster | Significant (t=45.65)  |
+-------------------------------+------------+------------+--------------+------------------------+
| regex_effbot                  | 3.36 ms    | 3.23 ms    | 1.04x faster | Significant (t=7.84)   |
+-------------------------------+------------+------------+--------------+------------------------+
| regex_v8                      | 27.6 ms    | 24.7 ms    | 1.12x faster | Significant (t=51.00)  |
+-------------------------------+------------+------------+--------------+------------------------+
| richards                      | 85.1 ms    | 56.0 ms    | 1.52x faster | Significant (t=84.28)  |
+-------------------------------+------------+------------+--------------+------------------------+
| richards_super                | 104 ms     | 67.5 ms    | 1.54x faster | Significant (t=81.27)  |
+-------------------------------+------------+------------+--------------+------------------------+
| scimark_fft                   | 485 ms     | 387 ms     | 1.25x faster | Significant (t=102.78) |
+-------------------------------+------------+------------+--------------+------------------------+
| scimark_lu                    | 197 ms     | 141 ms     | 1.39x faster | Significant (t=71.19)  |
+-------------------------------+------------+------------+--------------+------------------------+
| scimark_monte_carlo           | 128 ms     | 77.1 ms    | 1.66x faster | Significant (t=127.37) |
+-------------------------------+------------+------------+--------------+------------------------+
| scimark_sor                   | 232 ms     | 140 ms     | 1.66x faster | Significant (t=112.82) |
+-------------------------------+------------+------------+--------------+------------------------+
| scimark_sparse_mat_mult       | 6.67 ms    | 4.90 ms    | 1.36x faster | Significant (t=30.12)  |
+-------------------------------+------------+------------+--------------+------------------------+
| spectral_norm                 | 164 ms     | 131 ms     | 1.25x faster | Significant (t=52.68)  |
+-------------------------------+------------+------------+--------------+------------------------+
| sqlalchemy_declarative        | 161 ms     | 133 ms     | 1.21x faster | Significant (t=22.47)  |
+-------------------------------+------------+------------+--------------+------------------------+
| sqlalchemy_imperative         | 24.4 ms    | 21.1 ms    | 1.16x faster | Significant (t=33.55)  |
+-------------------------------+------------+------------+--------------+------------------------+
| sqlglot_normalize             | 158 ms     | 128 ms     | 1.23x faster | Significant (t=86.82)  |
+-------------------------------+------------+------------+--------------+------------------------+
| sqlglot_optimize              | 75.3 ms    | 61.3 ms    | 1.23x faster | Significant (t=89.97)  |
+-------------------------------+------------+------------+--------------+------------------------+
| sqlglot_parse                 | 2.33 ms    | 1.62 ms    | 1.44x faster | Significant (t=129.85) |
+-------------------------------+------------+------------+--------------+------------------------+
| sqlglot_transpile             | 2.76 ms    | 1.95 ms    | 1.41x faster | Significant (t=83.09)  |
+-------------------------------+------------+------------+--------------+------------------------+
| sqlite_synth                  | 3.68 us    | 3.07 us    | 1.20x faster | Significant (t=76.70)  |
+-------------------------------+------------+------------+--------------+------------------------+
| sympy_expand                  | 622 ms     | 537 ms     | 1.16x faster | Significant (t=29.76)  |
+-------------------------------+------------+------------+--------------+------------------------+
| sympy_integrate               | 27.0 ms    | 22.1 ms    | 1.22x faster | Significant (t=52.24)  |
+-------------------------------+------------+------------+--------------+------------------------+
| sympy_str                     | 373 ms     | 324 ms     | 1.15x faster | Significant (t=29.39)  |
+-------------------------------+------------+------------+--------------+------------------------+
| sympy_sum                     | 209 ms     | 181 ms     | 1.15x faster | Significant (t=43.71)  |
+-------------------------------+------------+------------+--------------+------------------------+
| telco                         | 8.34 ms    | 7.67 ms    | 1.09x faster | Significant (t=22.23)  |
+-------------------------------+------------+------------+--------------+------------------------+
| tomli_loads                   | 3.29 sec   | 2.50 sec   | 1.32x faster | Significant (t=66.74)  |
+-------------------------------+------------+------------+--------------+------------------------+
| tornado_http                  | 168 ms     | 137 ms     | 1.23x faster | Significant (t=27.00)  |
+-------------------------------+------------+------------+--------------+------------------------+
| typing_runtime_protocols      | 607 us     | 512 us     | 1.18x faster | Significant (t=30.73)  |
+-------------------------------+------------+------------+--------------+------------------------+
| unpack_sequence               | 58.4 ns    | 44.8 ns    | 1.30x faster | Significant (t=5.12)   |
+-------------------------------+------------+------------+--------------+------------------------+
| unpickle                      | 15.8 us    | 14.5 us    | 1.09x faster | Significant (t=18.21)  |
+-------------------------------+------------+------------+--------------+------------------------+
| unpickle_list                 | 5.39 us    | 5.20 us    | 1.04x faster | Significant (t=9.47)   |
+-------------------------------+------------+------------+--------------+------------------------+
| unpickle_pure_python          | 360 us     | 277 us     | 1.30x faster | Significant (t=79.85)  |
+-------------------------------+------------+------------+--------------+------------------------+
| xml_etree_generate            | 109 ms     | 91.8 ms    | 1.19x faster | Significant (t=70.86)  |
+-------------------------------+------------+------------+--------------+------------------------+
| xml_etree_iterparse           | 119 ms     | 110 ms     | 1.08x faster | Significant (t=29.01)  |
+-------------------------------+------------+------------+--------------+------------------------+
| xml_etree_parse               | 166 ms     | 165 ms     | 1.01x faster | Not significant        |
+-------------------------------+------------+------------+--------------+------------------------+
| xml_etree_process             | 87.5 ms    | 64.5 ms    | 1.36x faster | Significant (t=144.92) |
+-------------------------------+------------+------------+--------------+------------------------+

Skipped 8 benchmarks only in py311.json: async_tree_cpu_io_mixed_tg, async_tree_eager_cpu_io_mixed_tg, async_tree_eager_io_tg, async_tree_eager_memoization_tg, async_tree_eager_tg, async_tree_io_tg, async_tree_memoization_tg, async_tree_none_tg
    \end{minted}
    \caption{Comparison table of PyPerformance benchmark results between CPython versions 3.10.17 and 3.11.12.}
    \label{listing:pyperformance-results-310-311}
\end{code}

\begin{code}
    \begin{minted}[fontsize=\scriptsize]{text}
py311.json
==========

Performance version: 1.11.0
Report on Linux-6.8.0-1029-aws-x86_64-with-glibc2.39
Number of logical CPUs: 16
Start date: 2025-06-02 22:42:35.309501
End date: 2025-06-02 23:09:34.936710

py313.json
==========

Performance version: 1.11.0
Report on Linux-6.8.0-1029-aws-x86_64-with-glibc2.39
Number of logical CPUs: 16
Start date: 2025-06-02 22:07:57.491156
End date: 2025-06-02 22:31:36.629262

+----------------------------------+------------+------------+--------------+-------------------------+
| Benchmark                        | py311.json | py313.json | Change       | Significance            |
+==================================+============+============+==============+=========================+
| async_generators                 | 405 ms     | 460 ms     | 1.14x slower | Significant (t=-27.58)  |
+----------------------------------+------------+------------+--------------+-------------------------+
| async_tree_cpu_io_mixed          | 959 ms     | 677 ms     | 1.42x faster | Significant (t=106.37)  |
+----------------------------------+------------+------------+--------------+-------------------------+
| async_tree_cpu_io_mixed_tg       | 829 ms     | 742 ms     | 1.12x faster | Significant (t=20.80)   |
+----------------------------------+------------+------------+--------------+-------------------------+
| async_tree_eager                 | 607 ms     | 135 ms     | 4.51x faster | Significant (t=596.51)  |
+----------------------------------+------------+------------+--------------+-------------------------+
| async_tree_eager_cpu_io_mixed    | 959 ms     | 468 ms     | 2.05x faster | Significant (t=134.71)  |
+----------------------------------+------------+------------+--------------+-------------------------+
| async_tree_eager_cpu_io_mixed_tg | 827 ms     | 627 ms     | 1.32x faster | Significant (t=36.13)   |
+----------------------------------+------------+------------+--------------+-------------------------+
| async_tree_eager_io              | 1.39 sec   | 1.00 sec   | 1.39x faster | Significant (t=32.35)   |
+----------------------------------+------------+------------+--------------+-------------------------+
| async_tree_eager_io_tg           | 1.30 sec   | 1.10 sec   | 1.19x faster | Significant (t=11.67)   |
+----------------------------------+------------+------------+--------------+-------------------------+
| async_tree_eager_memoization     | 753 ms     | 281 ms     | 2.68x faster | Significant (t=301.57)  |
+----------------------------------+------------+------------+--------------+-------------------------+
| async_tree_eager_memoization_tg  | 694 ms     | 438 ms     | 1.58x faster | Significant (t=34.92)   |
+----------------------------------+------------+------------+--------------+-------------------------+
| async_tree_eager_tg              | 528 ms     | 318 ms     | 1.66x faster | Significant (t=56.23)   |
+----------------------------------+------------+------------+--------------+-------------------------+
| async_tree_io                    | 1.39 sec   | 966 ms     | 1.44x faster | Significant (t=36.45)   |
+----------------------------------+------------+------------+--------------+-------------------------+
| async_tree_io_tg                 | 1.30 sec   | 1.00 sec   | 1.30x faster | Significant (t=68.18)   |
+----------------------------------+------------+------------+--------------+-------------------------+
| async_tree_memoization           | 754 ms     | 536 ms     | 1.41x faster | Significant (t=16.34)   |
+----------------------------------+------------+------------+--------------+-------------------------+
| async_tree_memoization_tg        | 700 ms     | 533 ms     | 1.31x faster | Significant (t=24.40)   |
+----------------------------------+------------+------------+--------------+-------------------------+
| async_tree_none                  | 607 ms     | 412 ms     | 1.47x faster | Significant (t=144.13)  |
+----------------------------------+------------+------------+--------------+-------------------------+
| async_tree_none_tg               | 528 ms     | 381 ms     | 1.39x faster | Significant (t=105.89)  |
+----------------------------------+------------+------------+--------------+-------------------------+
| asyncio_tcp                      | 775 ms     | 476 ms     | 1.63x faster | Significant (t=124.10)  |
+----------------------------------+------------+------------+--------------+-------------------------+
| asyncio_tcp_ssl                  | 3.63 sec   | 1.74 sec   | 2.08x faster | Significant (t=853.90)  |
+----------------------------------+------------+------------+--------------+-------------------------+
| asyncio_websockets               | 633 ms     | 642 ms     | 1.01x slower | Not significant         |
+----------------------------------+------------+------------+--------------+-------------------------+
| bench_mp_pool                    | 12.8 ms    | 12.7 ms    | 1.01x faster | Not significant         |
+----------------------------------+------------+------------+--------------+-------------------------+
| bench_thread_pool                | 1.69 ms    | 1.68 ms    | 1.01x faster | Not significant         |
+----------------------------------+------------+------------+--------------+-------------------------+
| chameleon                        | 7.88 ms    | 8.15 ms    | 1.03x slower | Significant (t=-11.58)  |
+----------------------------------+------------+------------+--------------+-------------------------+
| chaos                            | 81.1 ms    | 71.5 ms    | 1.13x faster | Significant (t=23.56)   |
+----------------------------------+------------+------------+--------------+-------------------------+
| comprehensions                   | 25.5 us    | 20.2 us    | 1.26x faster | Significant (t=38.57)   |
+----------------------------------+------------+------------+--------------+-------------------------+
| coroutines                       | 28.8 ms    | 26.9 ms    | 1.07x faster | Significant (t=27.12)   |
+----------------------------------+------------+------------+--------------+-------------------------+
| coverage                         | 84.8 ms    | 108 ms     | 1.27x slower | Significant (t=-31.89)  |
+----------------------------------+------------+------------+--------------+-------------------------+
| create_gc_cycles                 | 1.11 ms    | 1.23 ms    | 1.11x slower | Significant (t=-21.36)  |
+----------------------------------+------------+------------+--------------+-------------------------+
| crypto_pyaes                     | 85.5 ms    | 78.7 ms    | 1.09x faster | Significant (t=26.47)   |
+----------------------------------+------------+------------+--------------+-------------------------+
| dask                             | 454 ms     | 449 ms     | 1.01x faster | Not significant         |
+----------------------------------+------------+------------+--------------+-------------------------+
| deepcopy                         | 407 us     | 449 us     | 1.11x slower | Significant (t=-49.99)  |
+----------------------------------+------------+------------+--------------+-------------------------+
| deepcopy_memo                    | 45.5 us    | 49.4 us    | 1.08x slower | Significant (t=-38.52)  |
+----------------------------------+------------+------------+--------------+-------------------------+
| deepcopy_reduce                  | 3.59 us    | 4.05 us    | 1.13x slower | Significant (t=-29.35)  |
+----------------------------------+------------+------------+--------------+-------------------------+
| deltablue                        | 4.08 ms    | 3.55 ms    | 1.15x faster | Significant (t=57.73)   |
+----------------------------------+------------+------------+--------------+-------------------------+
| django_template                  | 38.9 ms    | 40.9 ms    | 1.05x slower | Significant (t=-8.99)   |
+----------------------------------+------------+------------+--------------+-------------------------+
| docutils                         | 2.70 sec   | 2.74 sec   | 1.01x slower | Not significant         |
+----------------------------------+------------+------------+--------------+-------------------------+
| dulwich_log                      | 79.0 ms    | 79.3 ms    | 1.00x slower | Not significant         |
+----------------------------------+------------+------------+--------------+-------------------------+
| fannkuch                         | 434 ms     | 470 ms     | 1.08x slower | Significant (t=-11.43)  |
+----------------------------------+------------+------------+--------------+-------------------------+
| float                            | 86.2 ms    | 95.1 ms    | 1.10x slower | Significant (t=-31.15)  |
+----------------------------------+------------+------------+--------------+-------------------------+
| gc_traversal                     | 3.56 ms    | 3.75 ms    | 1.05x slower | Significant (t=-4.61)   |
+----------------------------------+------------+------------+--------------+-------------------------+
| generators                       | 55.7 ms    | 36.9 ms    | 1.51x faster | Significant (t=122.22)  |
+----------------------------------+------------+------------+--------------+-------------------------+
| genshi_text                      | 25.6 ms    | 26.3 ms    | 1.03x slower | Significant (t=-4.97)   |
+----------------------------------+------------+------------+--------------+-------------------------+
| genshi_xml                       | 61.1 ms    | 61.2 ms    | 1.00x slower | Not significant         |
+----------------------------------+------------+------------+--------------+-------------------------+
| go                               | 166 ms     | 171 ms     | 1.03x slower | Significant (t=-8.53)   |
+----------------------------------+------------+------------+--------------+-------------------------+
| hexiom                           | 7.43 ms    | 6.89 ms    | 1.08x faster | Significant (t=25.05)   |
+----------------------------------+------------+------------+--------------+-------------------------+
| html5lib                         | 81.4 ms    | 85.1 ms    | 1.05x slower | Significant (t=-4.97)   |
+----------------------------------+------------+------------+--------------+-------------------------+
| json_dumps                       | 13.6 ms    | 12.0 ms    | 1.14x faster | Significant (t=44.50)   |
+----------------------------------+------------+------------+--------------+-------------------------+
| json_loads                       | 28.4 us    | 30.3 us    | 1.07x slower | Significant (t=-23.47)  |
+----------------------------------+------------+------------+--------------+-------------------------+
| logging_format                   | 7.91 us    | 7.79 us    | 1.02x faster | Not significant         |
+----------------------------------+------------+------------+--------------+-------------------------+
| logging_silent                   | 123 ns     | 137 ns     | 1.11x slower | Significant (t=-14.30)  |
+----------------------------------+------------+------------+--------------+-------------------------+
| logging_simple                   | 7.10 us    | 6.75 us    | 1.05x faster | Significant (t=7.92)    |
+----------------------------------+------------+------------+--------------+-------------------------+
| mako                             | 11.7 ms    | 12.8 ms    | 1.09x slower | Significant (t=-30.48)  |
+----------------------------------+------------+------------+--------------+-------------------------+
| mdp                              | 3.20 sec   | 3.02 sec   | 1.06x faster | Significant (t=9.91)    |
+----------------------------------+------------+------------+--------------+-------------------------+
| meteor_contest                   | 109 ms     | 116 ms     | 1.07x slower | Significant (t=-19.41)  |
+----------------------------------+------------+------------+--------------+-------------------------+
| nbody                            | 102 ms     | 106 ms     | 1.04x slower | Significant (t=-12.11)  |
+----------------------------------+------------+------------+--------------+-------------------------+
| nqueens                          | 94.7 ms    | 92.6 ms    | 1.02x faster | Significant (t=4.58)    |
+----------------------------------+------------+------------+--------------+-------------------------+
| pathlib                          | 27.6 ms    | 26.0 ms    | 1.06x faster | Significant (t=13.29)   |
+----------------------------------+------------+------------+--------------+-------------------------+
| pickle                           | 11.8 us    | 14.9 us    | 1.27x slower | Significant (t=-54.21)  |
+----------------------------------+------------+------------+--------------+-------------------------+
| pickle_dict                      | 29.5 us    | 33.3 us    | 1.13x slower | Significant (t=-43.03)  |
+----------------------------------+------------+------------+--------------+-------------------------+
| pickle_list                      | 4.16 us    | 4.98 us    | 1.20x slower | Significant (t=-39.29)  |
+----------------------------------+------------+------------+--------------+-------------------------+
| pickle_pure_python               | 364 us     | 359 us     | 1.01x faster | Not significant         |
+----------------------------------+------------+------------+--------------+-------------------------+
| pidigits                         | 208 ms     | 218 ms     | 1.05x slower | Significant (t=-34.21)  |
+----------------------------------+------------+------------+--------------+-------------------------+
| pprint_pformat                   | 1.70 sec   | 1.82 sec   | 1.07x slower | Significant (t=-14.43)  |
+----------------------------------+------------+------------+--------------+-------------------------+
| pprint_safe_repr                 | 825 ms     | 884 ms     | 1.07x slower | Significant (t=-22.24)  |
+----------------------------------+------------+------------+--------------+-------------------------+
| pyflate                          | 463 ms     | 513 ms     | 1.11x slower | Significant (t=-25.73)  |
+----------------------------------+------------+------------+--------------+-------------------------+
| python_startup                   | 10.2 ms    | 12.5 ms    | 1.22x slower | Significant (t=-292.44) |
+----------------------------------+------------+------------+--------------+-------------------------+
| python_startup_no_site           | 7.62 ms    | 8.57 ms    | 1.12x slower | Significant (t=-136.29) |
+----------------------------------+------------+------------+--------------+-------------------------+
| raytrace                         | 355 ms     | 309 ms     | 1.15x faster | Significant (t=43.55)   |
+----------------------------------+------------+------------+--------------+-------------------------+
| regex_compile                    | 161 ms     | 156 ms     | 1.03x faster | Significant (t=10.62)   |
+----------------------------------+------------+------------+--------------+-------------------------+
| regex_dna                        | 182 ms     | 215 ms     | 1.18x slower | Significant (t=-38.47)  |
+----------------------------------+------------+------------+--------------+-------------------------+
| regex_effbot                     | 3.23 ms    | 3.24 ms    | 1.00x slower | Not significant         |
+----------------------------------+------------+------------+--------------+-------------------------+
| regex_v8                         | 24.7 ms    | 27.9 ms    | 1.13x slower | Significant (t=-27.49)  |
+----------------------------------+------------+------------+--------------+-------------------------+
| richards                         | 56.0 ms    | 60.7 ms    | 1.08x slower | Significant (t=-12.87)  |
+----------------------------------+------------+------------+--------------+-------------------------+
| richards_super                   | 67.5 ms    | 67.5 ms    | 1.00x slower | Not significant         |
+----------------------------------+------------+------------+--------------+-------------------------+
| scimark_fft                      | 387 ms     | 419 ms     | 1.08x slower | Significant (t=-18.47)  |
+----------------------------------+------------+------------+--------------+-------------------------+
| scimark_lu                       | 141 ms     | 137 ms     | 1.03x faster | Significant (t=8.59)    |
+----------------------------------+------------+------------+--------------+-------------------------+
| scimark_monte_carlo              | 77.1 ms    | 78.9 ms    | 1.02x slower | Significant (t=-8.31)   |
+----------------------------------+------------+------------+--------------+-------------------------+
| scimark_sor                      | 140 ms     | 157 ms     | 1.12x slower | Significant (t=-62.46)  |
+----------------------------------+------------+------------+--------------+-------------------------+
| scimark_sparse_mat_mult          | 4.90 ms    | 5.50 ms    | 1.12x slower | Significant (t=-16.82)  |
+----------------------------------+------------+------------+--------------+-------------------------+
| spectral_norm                    | 131 ms     | 128 ms     | 1.02x faster | Not significant         |
+----------------------------------+------------+------------+--------------+-------------------------+
| sqlglot_normalize                | 128 ms     | 129 ms     | 1.01x slower | Not significant         |
+----------------------------------+------------+------------+--------------+-------------------------+
| sqlglot_optimize                 | 61.3 ms    | 62.6 ms    | 1.02x slower | Significant (t=-6.14)   |
+----------------------------------+------------+------------+--------------+-------------------------+
| sqlglot_parse                    | 1.62 ms    | 1.49 ms    | 1.08x faster | Significant (t=43.70)   |
+----------------------------------+------------+------------+--------------+-------------------------+
| sqlglot_transpile                | 1.95 ms    | 1.82 ms    | 1.07x faster | Significant (t=24.15)   |
+----------------------------------+------------+------------+--------------+-------------------------+
| sqlite_synth                     | 3.07 us    | 3.37 us    | 1.10x slower | Significant (t=-34.84)  |
+----------------------------------+------------+------------+--------------+-------------------------+
| sympy_expand                     | 537 ms     | 523 ms     | 1.03x faster | Significant (t=6.58)    |
+----------------------------------+------------+------------+--------------+-------------------------+
| sympy_integrate                  | 22.1 ms    | 21.2 ms    | 1.04x faster | Significant (t=18.00)   |
+----------------------------------+------------+------------+--------------+-------------------------+
| sympy_str                        | 324 ms     | 307 ms     | 1.06x faster | Significant (t=13.82)   |
+----------------------------------+------------+------------+--------------+-------------------------+
| sympy_sum                        | 181 ms     | 162 ms     | 1.12x faster | Significant (t=31.73)   |
+----------------------------------+------------+------------+--------------+-------------------------+
| telco                            | 7.67 ms    | 9.23 ms    | 1.20x slower | Significant (t=-39.84)  |
+----------------------------------+------------+------------+--------------+-------------------------+
| tomli_loads                      | 2.50 sec   | 2.50 sec   | 1.00x slower | Not significant         |
+----------------------------------+------------+------------+--------------+-------------------------+
| tornado_http                     | 137 ms     | 134 ms     | 1.02x faster | Not significant         |
+----------------------------------+------------+------------+--------------+-------------------------+
| typing_runtime_protocols         | 512 us     | 189 us     | 2.71x faster | Significant (t=186.48)  |
+----------------------------------+------------+------------+--------------+-------------------------+
| unpack_sequence                  | 44.8 ns    | 52.9 ns    | 1.18x slower | Significant (t=-3.27)   |
+----------------------------------+------------+------------+--------------+-------------------------+
| unpickle                         | 14.5 us    | 16.0 us    | 1.10x slower | Significant (t=-19.09)  |
+----------------------------------+------------+------------+--------------+-------------------------+
| unpickle_list                    | 5.20 us    | 5.64 us    | 1.09x slower | Significant (t=-31.35)  |
+----------------------------------+------------+------------+--------------+-------------------------+
| unpickle_pure_python             | 277 us     | 262 us     | 1.05x faster | Significant (t=18.66)   |
+----------------------------------+------------+------------+--------------+-------------------------+
| xml_etree_generate               | 91.8 ms    | 99.3 ms    | 1.08x slower | Significant (t=-39.71)  |
+----------------------------------+------------+------------+--------------+-------------------------+
| xml_etree_iterparse              | 110 ms     | 112 ms     | 1.02x slower | Not significant         |
+----------------------------------+------------+------------+--------------+-------------------------+
| xml_etree_parse                  | 165 ms     | 164 ms     | 1.01x faster | Not significant         |
+----------------------------------+------------+------------+--------------+-------------------------+
| xml_etree_process                | 64.5 ms    | 68.7 ms    | 1.07x slower | Significant (t=-39.35)  |
+----------------------------------+------------+------------+--------------+-------------------------+

Skipped 2 benchmarks only in py311.json: sqlalchemy_declarative, sqlalchemy_imperative

Skipped 1 benchmarks only in py313.json: 2to3
    \end{minted}
    \caption{Comparison table of PyPerformance benchmark results between CPython versions 3.11.12 and 3.13.3.}
    \label{listing:pyperformance-results-311-313}
\end{code}

\begin{code}
    \begin{minted}[fontsize=\scriptsize]{text}
py313.json
==========

Performance version: 1.11.0
Report on Linux-6.8.0-1029-aws-x86_64-with-glibc2.39
Number of logical CPUs: 16
Start date: 2025-06-02 22:07:57.491156
End date: 2025-06-02 22:31:36.629262

py313-jit.json
==============

Performance version: 1.11.0
Report on Linux-6.8.0-1029-aws-x86_64-with-glibc2.39
Number of logical CPUs: 16
Start date: 2025-06-02 21:40:29.802724
End date: 2025-06-02 22:04:08.722964

+----------------------------------+------------+----------------+--------------+-------------------------+
| Benchmark                        | py313.json | py313-jit.json | Change       | Significance            |
+==================================+============+================+==============+=========================+
| async_generators                 | 460 ms     | 485 ms         | 1.05x slower | Significant (t=-8.31)   |
+----------------------------------+------------+----------------+--------------+-------------------------+
| async_tree_cpu_io_mixed          | 677 ms     | 672 ms         | 1.01x faster | Not significant         |
+----------------------------------+------------+----------------+--------------+-------------------------+
| async_tree_cpu_io_mixed_tg       | 742 ms     | 735 ms         | 1.01x faster | Not significant         |
+----------------------------------+------------+----------------+--------------+-------------------------+
| async_tree_eager                 | 135 ms     | 138 ms         | 1.03x slower | Significant (t=-5.82)   |
+----------------------------------+------------+----------------+--------------+-------------------------+
| async_tree_eager_cpu_io_mixed    | 468 ms     | 465 ms         | 1.01x faster | Not significant         |
+----------------------------------+------------+----------------+--------------+-------------------------+
| async_tree_eager_cpu_io_mixed_tg | 627 ms     | 620 ms         | 1.01x faster | Not significant         |
+----------------------------------+------------+----------------+--------------+-------------------------+
| async_tree_eager_io              | 1.00 sec   | 997 ms         | 1.00x faster | Not significant         |
+----------------------------------+------------+----------------+--------------+-------------------------+
| async_tree_eager_io_tg           | 1.10 sec   | 1.12 sec       | 1.02x slower | Not significant         |
+----------------------------------+------------+----------------+--------------+-------------------------+
| async_tree_eager_memoization     | 281 ms     | 289 ms         | 1.03x slower | Significant (t=-3.93)   |
+----------------------------------+------------+----------------+--------------+-------------------------+
| async_tree_eager_memoization_tg  | 438 ms     | 439 ms         | 1.00x slower | Not significant         |
+----------------------------------+------------+----------------+--------------+-------------------------+
| async_tree_eager_tg              | 318 ms     | 322 ms         | 1.01x slower | Not significant         |
+----------------------------------+------------+----------------+--------------+-------------------------+
| async_tree_io                    | 966 ms     | 960 ms         | 1.01x faster | Not significant         |
+----------------------------------+------------+----------------+--------------+-------------------------+
| async_tree_io_tg                 | 1.00 sec   | 990 ms         | 1.01x faster | Not significant         |
+----------------------------------+------------+----------------+--------------+-------------------------+
| async_tree_memoization           | 536 ms     | 538 ms         | 1.00x slower | Not significant         |
+----------------------------------+------------+----------------+--------------+-------------------------+
| async_tree_memoization_tg        | 533 ms     | 531 ms         | 1.00x faster | Not significant         |
+----------------------------------+------------+----------------+--------------+-------------------------+
| async_tree_none                  | 412 ms     | 414 ms         | 1.01x slower | Not significant         |
+----------------------------------+------------+----------------+--------------+-------------------------+
| async_tree_none_tg               | 381 ms     | 383 ms         | 1.00x slower | Not significant         |
+----------------------------------+------------+----------------+--------------+-------------------------+
| asyncio_tcp                      | 476 ms     | 504 ms         | 1.06x slower | Significant (t=-10.55)  |
+----------------------------------+------------+----------------+--------------+-------------------------+
| asyncio_tcp_ssl                  | 1.74 sec   | 1.75 sec       | 1.00x slower | Not significant         |
+----------------------------------+------------+----------------+--------------+-------------------------+
| asyncio_websockets               | 642 ms     | 647 ms         | 1.01x slower | Not significant         |
+----------------------------------+------------+----------------+--------------+-------------------------+
| bench_mp_pool                    | 12.7 ms    | 13.1 ms        | 1.03x slower | Not significant         |
+----------------------------------+------------+----------------+--------------+-------------------------+
| bench_thread_pool                | 1.68 ms    | 1.74 ms        | 1.04x slower | Significant (t=-13.58)  |
+----------------------------------+------------+----------------+--------------+-------------------------+
| chameleon                        | 8.15 ms    | 8.22 ms        | 1.01x slower | Not significant         |
+----------------------------------+------------+----------------+--------------+-------------------------+
| chaos                            | 71.5 ms    | 71.2 ms        | 1.00x faster | Not significant         |
+----------------------------------+------------+----------------+--------------+-------------------------+
| comprehensions                   | 20.2 us    | 19.5 us        | 1.03x faster | Significant (t=4.31)    |
+----------------------------------+------------+----------------+--------------+-------------------------+
| coroutines                       | 26.9 ms    | 26.9 ms        | 1.00x faster | Not significant         |
+----------------------------------+------------+----------------+--------------+-------------------------+
| coverage                         | 108 ms     | 96.5 ms        | 1.12x faster | Significant (t=23.68)   |
+----------------------------------+------------+----------------+--------------+-------------------------+
| create_gc_cycles                 | 1.23 ms    | 1.24 ms        | 1.01x slower | Not significant         |
+----------------------------------+------------+----------------+--------------+-------------------------+
| crypto_pyaes                     | 78.7 ms    | 76.2 ms        | 1.03x faster | Significant (t=5.83)    |
+----------------------------------+------------+----------------+--------------+-------------------------+
| dask                             | 449 ms     | 455 ms         | 1.01x slower | Not significant         |
+----------------------------------+------------+----------------+--------------+-------------------------+
| deepcopy                         | 449 us     | 466 us         | 1.04x slower | Significant (t=-4.65)   |
+----------------------------------+------------+----------------+--------------+-------------------------+
| deepcopy_memo                    | 49.4 us    | 45.5 us        | 1.09x faster | Significant (t=40.04)   |
+----------------------------------+------------+----------------+--------------+-------------------------+
| deepcopy_reduce                  | 4.05 us    | 4.11 us        | 1.01x slower | Not significant         |
+----------------------------------+------------+----------------+--------------+-------------------------+
| deltablue                        | 3.55 ms    | 3.97 ms        | 1.12x slower | Significant (t=-55.41)  |
+----------------------------------+------------+----------------+--------------+-------------------------+
| django_template                  | 40.9 ms    | 43.3 ms        | 1.06x slower | Significant (t=-7.95)   |
+----------------------------------+------------+----------------+--------------+-------------------------+
| docutils                         | 2.74 sec   | 2.87 sec       | 1.05x slower | Significant (t=-8.39)   |
+----------------------------------+------------+----------------+--------------+-------------------------+
| dulwich_log                      | 79.3 ms    | 81.1 ms        | 1.02x slower | Significant (t=-2.44)   |
+----------------------------------+------------+----------------+--------------+-------------------------+
| fannkuch                         | 470 ms     | 423 ms         | 1.11x faster | Significant (t=24.08)   |
+----------------------------------+------------+----------------+--------------+-------------------------+
| float                            | 95.1 ms    | 82.4 ms        | 1.15x faster | Significant (t=37.45)   |
+----------------------------------+------------+----------------+--------------+-------------------------+
| gc_traversal                     | 3.75 ms    | 3.76 ms        | 1.00x slower | Not significant         |
+----------------------------------+------------+----------------+--------------+-------------------------+
| generators                       | 36.9 ms    | 37.1 ms        | 1.00x slower | Not significant         |
+----------------------------------+------------+----------------+--------------+-------------------------+
| genshi_text                      | 26.3 ms    | 30.2 ms        | 1.15x slower | Significant (t=-22.09)  |
+----------------------------------+------------+----------------+--------------+-------------------------+
| genshi_xml                       | 61.2 ms    | 68.5 ms        | 1.12x slower | Significant (t=-32.32)  |
+----------------------------------+------------+----------------+--------------+-------------------------+
| go                               | 171 ms     | 174 ms         | 1.02x slower | Significant (t=-5.09)   |
+----------------------------------+------------+----------------+--------------+-------------------------+
| hexiom                           | 6.89 ms    | 7.13 ms        | 1.04x slower | Significant (t=-8.44)   |
+----------------------------------+------------+----------------+--------------+-------------------------+
| html5lib                         | 85.1 ms    | 82.9 ms        | 1.03x faster | Significant (t=9.55)    |
+----------------------------------+------------+----------------+--------------+-------------------------+
| json_dumps                       | 12.0 ms    | 11.8 ms        | 1.01x faster | Not significant         |
+----------------------------------+------------+----------------+--------------+-------------------------+
| json_loads                       | 30.3 us    | 30.4 us        | 1.00x slower | Not significant         |
+----------------------------------+------------+----------------+--------------+-------------------------+
| logging_format                   | 7.79 us    | 7.25 us        | 1.07x faster | Significant (t=7.00)    |
+----------------------------------+------------+----------------+--------------+-------------------------+
| logging_silent                   | 137 ns     | 135 ns         | 1.01x faster | Not significant         |
+----------------------------------+------------+----------------+--------------+-------------------------+
| logging_simple                   | 6.75 us    | 6.65 us        | 1.02x faster | Not significant         |
+----------------------------------+------------+----------------+--------------+-------------------------+
| mako                             | 12.8 ms    | 11.6 ms        | 1.10x faster | Significant (t=12.42)   |
+----------------------------------+------------+----------------+--------------+-------------------------+
| mdp                              | 3.02 sec   | 3.06 sec       | 1.01x slower | Not significant         |
+----------------------------------+------------+----------------+--------------+-------------------------+
| meteor_contest                   | 116 ms     | 117 ms         | 1.00x slower | Not significant         |
+----------------------------------+------------+----------------+--------------+-------------------------+
| nbody                            | 106 ms     | 113 ms         | 1.06x slower | Significant (t=-14.11)  |
+----------------------------------+------------+----------------+--------------+-------------------------+
| nqueens                          | 92.6 ms    | 104 ms         | 1.12x slower | Significant (t=-18.68)  |
+----------------------------------+------------+----------------+--------------+-------------------------+
| pathlib                          | 26.0 ms    | 26.1 ms        | 1.00x slower | Not significant         |
+----------------------------------+------------+----------------+--------------+-------------------------+
| pickle                           | 14.9 us    | 14.7 us        | 1.01x faster | Not significant         |
+----------------------------------+------------+----------------+--------------+-------------------------+
| pickle_dict                      | 33.3 us    | 32.6 us        | 1.02x faster | Significant (t=4.89)    |
+----------------------------------+------------+----------------+--------------+-------------------------+
| pickle_list                      | 4.98 us    | 4.97 us        | 1.00x faster | Not significant         |
+----------------------------------+------------+----------------+--------------+-------------------------+
| pickle_pure_python               | 359 us     | 367 us         | 1.02x slower | Significant (t=-10.93)  |
+----------------------------------+------------+----------------+--------------+-------------------------+
| pidigits                         | 218 ms     | 212 ms         | 1.03x faster | Significant (t=32.66)   |
+----------------------------------+------------+----------------+--------------+-------------------------+
| pprint_pformat                   | 1.82 sec   | 1.81 sec       | 1.00x faster | Not significant         |
+----------------------------------+------------+----------------+--------------+-------------------------+
| pprint_safe_repr                 | 884 ms     | 886 ms         | 1.00x slower | Not significant         |
+----------------------------------+------------+----------------+--------------+-------------------------+
| pyflate                          | 513 ms     | 482 ms         | 1.06x faster | Significant (t=13.39)   |
+----------------------------------+------------+----------------+--------------+-------------------------+
| python_startup                   | 12.5 ms    | 14.3 ms        | 1.15x slower | Significant (t=-169.42) |
+----------------------------------+------------+----------------+--------------+-------------------------+
| python_startup_no_site           | 8.57 ms    | 10.4 ms        | 1.22x slower | Significant (t=-303.79) |
+----------------------------------+------------+----------------+--------------+-------------------------+
| raytrace                         | 309 ms     | 318 ms         | 1.03x slower | Significant (t=-9.91)   |
+----------------------------------+------------+----------------+--------------+-------------------------+
| regex_compile                    | 156 ms     | 153 ms         | 1.02x faster | Significant (t=7.49)    |
+----------------------------------+------------+----------------+--------------+-------------------------+
| regex_dna                        | 215 ms     | 199 ms         | 1.08x faster | Significant (t=20.71)   |
+----------------------------------+------------+----------------+--------------+-------------------------+
| regex_effbot                     | 3.24 ms    | 3.13 ms        | 1.04x faster | Significant (t=4.77)    |
+----------------------------------+------------+----------------+--------------+-------------------------+
| regex_v8                         | 27.9 ms    | 28.0 ms        | 1.00x slower | Not significant         |
+----------------------------------+------------+----------------+--------------+-------------------------+
| richards                         | 60.7 ms    | 47.6 ms        | 1.27x faster | Significant (t=46.32)   |
+----------------------------------+------------+----------------+--------------+-------------------------+
| richards_super                   | 67.5 ms    | 55.4 ms        | 1.22x faster | Significant (t=40.93)   |
+----------------------------------+------------+----------------+--------------+-------------------------+
| scimark_fft                      | 419 ms     | 359 ms         | 1.17x faster | Significant (t=32.96)   |
+----------------------------------+------------+----------------+--------------+-------------------------+
| scimark_lu                       | 137 ms     | 157 ms         | 1.14x slower | Significant (t=-46.95)  |
+----------------------------------+------------+----------------+--------------+-------------------------+
| scimark_monte_carlo              | 78.9 ms    | 80.7 ms        | 1.02x slower | Significant (t=-7.75)   |
+----------------------------------+------------+----------------+--------------+-------------------------+
| scimark_sor                      | 157 ms     | 162 ms         | 1.03x slower | Significant (t=-15.13)  |
+----------------------------------+------------+----------------+--------------+-------------------------+
| scimark_sparse_mat_mult          | 5.50 ms    | 4.64 ms        | 1.19x faster | Significant (t=23.43)   |
+----------------------------------+------------+----------------+--------------+-------------------------+
| spectral_norm                    | 128 ms     | 107 ms         | 1.20x faster | Significant (t=41.15)   |
+----------------------------------+------------+----------------+--------------+-------------------------+
| sqlglot_normalize                | 129 ms     | 132 ms         | 1.03x slower | Significant (t=-4.35)   |
+----------------------------------+------------+----------------+--------------+-------------------------+
| sqlglot_optimize                 | 62.6 ms    | 65.2 ms        | 1.04x slower | Significant (t=-9.74)   |
+----------------------------------+------------+----------------+--------------+-------------------------+
| sqlglot_parse                    | 1.49 ms    | 1.46 ms        | 1.02x faster | Not significant         |
+----------------------------------+------------+----------------+--------------+-------------------------+
| sqlglot_transpile                | 1.82 ms    | 1.81 ms        | 1.01x faster | Not significant         |
+----------------------------------+------------+----------------+--------------+-------------------------+
| sqlite_synth                     | 3.37 us    | 3.27 us        | 1.03x faster | Significant (t=8.77)    |
+----------------------------------+------------+----------------+--------------+-------------------------+
| sympy_expand                     | 523 ms     | 574 ms         | 1.10x slower | Significant (t=-19.84)  |
+----------------------------------+------------+----------------+--------------+-------------------------+
| sympy_integrate                  | 21.2 ms    | 22.8 ms        | 1.07x slower | Significant (t=-31.11)  |
+----------------------------------+------------+----------------+--------------+-------------------------+
| sympy_str                        | 307 ms     | 327 ms         | 1.06x slower | Significant (t=-16.63)  |
+----------------------------------+------------+----------------+--------------+-------------------------+
| sympy_sum                        | 162 ms     | 175 ms         | 1.08x slower | Significant (t=-22.14)  |
+----------------------------------+------------+----------------+--------------+-------------------------+
| telco                            | 9.23 ms    | 9.47 ms        | 1.03x slower | Significant (t=-6.64)   |
+----------------------------------+------------+----------------+--------------+-------------------------+
| tomli_loads                      | 2.50 sec   | 2.27 sec       | 1.10x faster | Significant (t=28.24)   |
+----------------------------------+------------+----------------+--------------+-------------------------+
| tornado_http                     | 134 ms     | 139 ms         | 1.04x slower | Significant (t=-4.64)   |
+----------------------------------+------------+----------------+--------------+-------------------------+
| typing_runtime_protocols         | 189 us     | 199 us         | 1.05x slower | Significant (t=-6.47)   |
+----------------------------------+------------+----------------+--------------+-------------------------+
| unpack_sequence                  | 52.9 ns    | 197 ns         | 3.73x slower | Significant (t=-366.64) |
+----------------------------------+------------+----------------+--------------+-------------------------+
| unpickle                         | 16.0 us    | 15.9 us        | 1.01x faster | Not significant         |
+----------------------------------+------------+----------------+--------------+-------------------------+
| unpickle_list                    | 5.64 us    | 5.36 us        | 1.05x faster | Significant (t=12.69)   |
+----------------------------------+------------+----------------+--------------+-------------------------+
| unpickle_pure_python             | 262 us     | 260 us         | 1.01x faster | Not significant         |
+----------------------------------+------------+----------------+--------------+-------------------------+
| xml_etree_generate               | 99.3 ms    | 96.7 ms        | 1.03x faster | Significant (t=11.44)   |
+----------------------------------+------------+----------------+--------------+-------------------------+
| xml_etree_iterparse              | 112 ms     | 108 ms         | 1.03x faster | Significant (t=13.58)   |
+----------------------------------+------------+----------------+--------------+-------------------------+
| xml_etree_parse                  | 164 ms     | 164 ms         | 1.00x faster | Not significant         |
+----------------------------------+------------+----------------+--------------+-------------------------+
| xml_etree_process                | 68.7 ms    | 68.6 ms        | 1.00x faster | Not significant         |
+----------------------------------+------------+----------------+--------------+-------------------------+

Skipped 1 benchmarks only in py313.json: 2to3
    \end{minted}
    \caption{Comparison table of PyPerformance benchmark results between CPython 3.13.3 with and without the JIT enabled.}
    \label{listing:pyperformance-results-313-313-jit}
\end{code}

% \chapter{MLIR workloads}
% \label{chap:mlir-workloads}

% \section{Constant folding}

% % Hook
% % Argument
% % Link

% \vspace{2em}

% \begin{code}
%     \begin{minted}{python}
% def constant_folding_module(size: int) -> ModuleOp:
%     """Generate a constant folding workload of a given size.

%     The output of running the command
%     `print(WorkloadBuilder().constant_folding_module(size=5))` is shown
%     below:

%     ```mlir
%     "builtin.module"() ({
%         %0 = "arith.constant"() {"value" = 865 : i32} : () -> i32
%         %1 = "arith.constant"() {"value" = 395 : i32} : () -> i32
%         %2 = "arith.addi"(%1, %0) : (i32, i32) -> i32
%         %3 = "arith.constant"() {"value" = 777 : i32} : () -> i32
%         %4 = "arith.addi"(%3, %2) : (i32, i32) -> i32
%         %5 = "arith.constant"() {"value" = 912 : i32} : () -> i32
%         "test.op"(%4) : (i32) -> ()
%     }) : () -> ()
%     ```
%     """
%     assert size >= 0
%     random.seed(RANDOM_SEED)
%     ops: list[Operation] = []
%     ops.append(ConstantOp(IntegerAttr(random.randint(1, 1000), i32)))
%     for i in range(1, size + 1):
%         if i % 2 == 0:
%             ops.append(AddiOp(ops[i - 1], ops[i - 2]))
%         else:
%             ops.append(ConstantOp(IntegerAttr(random.randint(1, 1000), i32)))
%     ops.append(TestOp([ops[(size // 2) * 2]]))
%     return ModuleOp(ops)
%     \end{minted}
%     \caption{.}
%     \label{listing:constant-folding-workload}
% \end{code}

% \vspace{2em}

% % \begin{code}
% %     \begin{minted}{mlir}
% % builtin.module {
% %   %0 = arith.constant 865 : i32
% %   %1 = arith.constant 395 : i32
% %   %2 = arith.addi %1, %0 : i32
% %   %3 = arith.constant 777 : i32
% %   %4 = arith.addi %3, %2 : i32
% %   %5 = arith.constant 912 : i32
% %   %6 = arith.addi %5, %4 : i32
% %   %7 = arith.constant 431 : i32
% %   %8 = arith.addi %7, %6 : i32
% %   %9 = arith.constant 42 : i32
% %   %10 = arith.addi %9, %8 : i32
% %   %11 = arith.constant 266 : i32
% %   %12 = arith.addi %11, %10 : i32
% %   %13 = arith.constant 989 : i32
% %   %14 = arith.addi %13, %12 : i32
% %   %15 = arith.constant 524 : i32
% %   %16 = arith.addi %15, %14 : i32
% %   %17 = arith.constant 498 : i32
% %   %18 = arith.addi %17, %16 : i32
% %   %19 = arith.constant 415 : i32
% %   %20 = arith.addi %19, %18 : i32
% %   "test.op"(%20) : (i32) -> ()
% % }
% %     \end{minted}
% %     \caption{.}
% %     \label{listing:bash-xdsl-ubench-run}
% % \end{code}

% % \vspace{2em}

\chapter{MLIR benchmark results}
\label{chap:mlir-benchmark-results}

\section{Pipeline phase micro-benchmark results}
% Commands to run
% Table of results

\section{How Slow is MLIR micro-benchmark results}

% Hook
The following section describes the procedure and provides the raw results from ``How Slow is MLIR's'' micro-benchmarks.
% Argument
% Link

\vspace{2em}

\begin{code}
    \begin{minted}[fontsize=\footnotesize]{text}
        git clone https://github.com/EdmundGoodman/llvm-project-benchmarks --depth 1
        mkdir -p llvm-project-benchmarks/build/
        cd llvm-project-benchmarks/build
        cmake -G Ninja ../llvm \
           -DLLVM_ENABLE_PROJECTS=mlir \
           -DLLVM_TARGETS_TO_BUILD="host" \
           -DLLVM_ENABLE_BENCHMARKS=ON \
           -DCMAKE_BUILD_TYPE=Release \
           -DCMAKE_C_COMPILER=clang-18 -DCMAKE_CXX_COMPILER=clang++-18
       ./tools/mlir/unittests/Benchmarks/MLIR_IR_Benchmark
    \end{minted}
    \caption{Bash commands to downloads, compiler, and run the benchmarks from ``How Slow is MLIR''.}
    \label{listing:bash-mlir-ubench-run}
\end{code}

\vspace{2em}

\begin{code}
    \begin{minted}[fontsize=\scriptsize]{text}
2025-05-14T12:19:17+00:00
Running ./tools/mlir/unittests/Benchmarks/MLIR_IR_Benchmark
Run on (16 X 2799.99 MHz CPU s)
CPU Caches:
  L1 Data 32 KiB (x8)
  L1 Instruction 32 KiB (x8)
  L2 Unified 512 KiB (x8)
  L3 Unified 16384 KiB (x2)
Load Average: 0.02, 0.06, 0.02
------------------------------------------------------------------------------------------------------
Benchmark                                                            Time             CPU   Iterations
------------------------------------------------------------------------------------------------------
Analysis/symbolTable/10                                            258 ns          258 ns      2695939
Analysis/symbolTable/64                                           1280 ns         1280 ns       543255
Analysis/symbolTable/512                                         11790 ns        11788 ns        60690
Analysis/symbolTable/4096                                       224244 ns       224225 ns         3104
Analysis/symbolTable/32768                                     2313859 ns      2313755 ns          301
Analysis/symbolTable/262144                                   20582734 ns     20580290 ns           34
Analysis/symbolTable/2097152                                 207555204 ns    207544488 ns            3
Analysis/symbolTable/10000000                               1111381830 ns   1111329506 ns            1
Analysis/symbolTable_BigO                                       110.60 N        110.60 N
Analysis/symbolTable_RMS                                             6 %             6 %
AttributesBench/sameString/10                                    25400 ns        25395 ns        27578
AttributesBench/sameString/64                                    29777 ns        29770 ns        23165
AttributesBench/sameString/512                                   67519 ns        67508 ns        10391
AttributesBench/sameString/4096                                 368952 ns       368882 ns         1901
AttributesBench/sameString/32768                               2616887 ns      2616441 ns          268
AttributesBench/sameString/262144                             20386755 ns     20382261 ns           35
AttributesBench/sameString/2097152                           160800051 ns    160770866 ns            4
AttributesBench/sameString/10000000                          789746984 ns    789711748 ns            1
AttributesBench/sameString_BigO                                  78.88 N         78.87 N
AttributesBench/sameString_RMS                                       1 %             1 %
AttributesBench/newString/10                                     26939 ns        26938 ns        26077
AttributesBench/newString/64                                     38844 ns        38835 ns        18031
AttributesBench/newString/512                                   176931 ns       176919 ns         3973
AttributesBench/newString/4096                                 1362397 ns      1362302 ns          509
AttributesBench/newString/32768                               11228015 ns     11227408 ns           62
AttributesBench/newString/262144                             111172923 ns    111167549 ns            6
AttributesBench/newString/2097152                           1591535825 ns   1591347447 ns            1
AttributesBench/newString/10000000                          8887091339 ns   8886357582 ns            1
AttributesBench/newString_BigO                                  882.93 N        882.86 N
AttributesBench/newString_RMS                                        8 %             8 %
AttributesBench/sameStringNoThreading/10                         25093 ns        25091 ns        27714
AttributesBench/sameStringNoThreading/64                         29761 ns        29759 ns        23732
AttributesBench/sameStringNoThreading/512                        67391 ns        67385 ns        10772
AttributesBench/sameStringNoThreading/4096                      362256 ns       362239 ns         1985
AttributesBench/sameStringNoThreading/32768                    2653860 ns      2653773 ns          263
AttributesBench/sameStringNoThreading/262144                  20414147 ns     20412688 ns           34
AttributesBench/sameStringNoThreading/2097152                161449035 ns    161410387 ns            4
AttributesBench/sameStringNoThreading/10000000               812195579 ns    812055621 ns            1
AttributesBench/sameStringNoThreading_BigO                       81.04 N         81.02 N
AttributesBench/sameStringNoThreading_RMS                            2 %             2 %
AttributesBench/newStringNoThreading/10                          25899 ns        25894 ns        26880
AttributesBench/newStringNoThreading/64                          33426 ns        33420 ns        20941
AttributesBench/newStringNoThreading/512                        111491 ns       111464 ns         6322
AttributesBench/newStringNoThreading/4096                       799421 ns       799274 ns          871
AttributesBench/newStringNoThreading/32768                     6615350 ns      6615040 ns          107
AttributesBench/newStringNoThreading/262144                   60571886 ns     60568929 ns           10
AttributesBench/newStringNoThreading/2097152                1011572721 ns   1011488998 ns            1
AttributesBench/newStringNoThreading/10000000               5860333524 ns   5859865437 ns            1
AttributesBench/newStringNoThreading_BigO                       581.43 N        581.38 N
AttributesBench/newStringNoThreading_RMS                             9 %             9 %
AttributesBench/sameStringMultithreaded/1                       177082 ns       132487 ns         5349
AttributesBench/sameStringMultithreaded/8                       188428 ns       138520 ns         4991
AttributesBench/sameStringMultithreaded/64                      217087 ns       158959 ns         4453
AttributesBench/sameStringMultithreaded/512                     401011 ns       186276 ns         3759
AttributesBench/sameStringMultithreaded/4096                   1692038 ns       197919 ns         3654
AttributesBench/sameStringMultithreaded/10000                  3872841 ns       204502 ns         1000
AttributesBench/sameStringMultithreaded_BigO                    391.99 N         25.31 N
AttributesBench/sameStringMultithreaded_RMS                         15 %            77 %
AttributesBench/newStringMultithreaded/1                        176590 ns       131450 ns         5397
AttributesBench/newStringMultithreaded/8                        225039 ns       158980 ns         4537
AttributesBench/newStringMultithreaded/64                       270241 ns       185683 ns         3781
AttributesBench/newStringMultithreaded/512                     1787474 ns       206312 ns         3342
AttributesBench/newStringMultithreaded/4096                   13267092 ns       219467 ns         1000
AttributesBench/newStringMultithreaded/10000                  31691415 ns       228920 ns          100
AttributesBench/newStringMultithreaded_BigO                    3179.93 N         28.25 N
AttributesBench/newStringMultithreaded_RMS                           2 %            77 %
AttributesBench/newStringEachMultithreaded/1                    181078 ns       134121 ns         5180
AttributesBench/newStringEachMultithreaded/8                    217968 ns       159165 ns         4343
AttributesBench/newStringEachMultithreaded/64                   393130 ns       197247 ns         3421
AttributesBench/newStringEachMultithreaded/512                 2220815 ns       212651 ns         3259
AttributesBench/newStringEachMultithreaded/4096               15542558 ns       238015 ns         1000
AttributesBench/newStringEachMultithreaded/10000              37028639 ns       262253 ns          100
AttributesBench/newStringEachMultithreaded_BigO                3717.53 N         31.79 N
AttributesBench/newStringEachMultithreaded_RMS                       2 %            75 %
AttributesBench/setAttrRaw/1                                       352 ns          352 ns      1894864
AttributesBench/setAttrRaw/8                                      1935 ns         1934 ns       362435
AttributesBench/setAttrRaw/64                                    14422 ns        14422 ns        48335
AttributesBench/setAttrRaw/512                                  114092 ns       114079 ns         6155
AttributesBench/setAttrRaw/4096                                 917976 ns       917885 ns          768
AttributesBench/setAttrRaw/32768                               7267919 ns      7267427 ns           96
AttributesBench/setAttrRaw/100000                             22257550 ns     22255369 ns           31
AttributesBench/setAttrRaw_BigO                                 222.50 N        222.48 N
AttributesBench/setAttrRaw_RMS                                       0 %             0 %
AttributesBench/setAttrProp/1                                     1.27 ns         1.27 ns    552856899
AttributesBench/setAttrProp/8                                     5.57 ns         5.57 ns    125431956
AttributesBench/setAttrProp/64                                    40.4 ns         40.3 ns     17349876
AttributesBench/setAttrProp/512                                    326 ns          326 ns      2139325
AttributesBench/setAttrProp/4096                                  2589 ns         2588 ns       270117
AttributesBench/setAttrProp/32768                                20657 ns        20655 ns        33859
AttributesBench/setAttrProp/100000                               62989 ns        62984 ns        11150
AttributesBench/setAttrProp_BigO                                  0.63 N          0.63 N
AttributesBench/setAttrProp_RMS                                      0 %             0 %
AttributesBench/setProp/1                                         1.27 ns         1.27 ns    527999624
AttributesBench/setProp/8                                         6.43 ns         6.43 ns    106786484
AttributesBench/setProp/64                                        41.7 ns         41.7 ns     16801887
AttributesBench/setProp/512                                        332 ns          332 ns      2105804
AttributesBench/setProp/4096                                      2591 ns         2591 ns       270559
AttributesBench/setProp/32768                                    20664 ns        20663 ns        33868
AttributesBench/setProp/262144                                  165411 ns       165400 ns         4230
AttributesBench/setProp/1000000                                 631175 ns       631128 ns         1109
AttributesBench/setProp_BigO                                      0.63 N          0.63 N
AttributesBench/setProp_RMS                                          0 %             0 %
AttributesBench/setPropHoist/1                                   0.947 ns        0.947 ns    740202993
AttributesBench/setPropHoist/8                                    5.05 ns         5.05 ns    138881195
AttributesBench/setPropHoist/64                                   40.3 ns         40.3 ns     17345436
AttributesBench/setPropHoist/512                                   325 ns          325 ns      2153624
AttributesBench/setPropHoist/4096                                 2586 ns         2585 ns       270515
AttributesBench/setPropHoist/32768                               20703 ns        20702 ns        33857
AttributesBench/setPropHoist/262144                             165570 ns       165564 ns         4225
AttributesBench/setPropHoist/1000000                            631529 ns       631485 ns         1112
AttributesBench/setPropHoist_BigO                                 0.63 N          0.63 N
AttributesBench/setPropHoist_RMS                                     0 %             0 %
ConstantFolding/folding/1                                         3895 ns         3832 ns       182475
ConstantFolding/folding/8                                        15109 ns        15069 ns        46046
ConstantFolding/folding/64                                      114518 ns       114466 ns         6091
ConstantFolding/folding/512                                     508223 ns       508155 ns         1390
ConstantFolding/folding/4096                                   3223717 ns      3223398 ns          214
ConstantFolding/folding/10000                                  7825265 ns      7824889 ns           90
ConstantFolding/folding_BigO                                    783.68 N        783.64 N
ConstantFolding/folding_RMS                                          3 %             3 %
ConstantFolding/opFolder/1                                        5910 ns         5866 ns       119213
ConstantFolding/opFolder/8                                       20958 ns        20932 ns        33807
ConstantFolding/opFolder/64                                     153989 ns       153958 ns         4553
ConstantFolding/opFolder/512                                    742573 ns       742498 ns          953
ConstantFolding/opFolder/4096                                  4734908 ns      4734590 ns          145
ConstantFolding/opFolder/10000                                11462842 ns     11462018 ns           63
ConstantFolding/opFolder_BigO                                  1148.40 N       1148.32 N
ConstantFolding/opFolder_RMS                                         3 %             3 %
ConstantFolding/llvm_folding/1                                    1456 ns         1381 ns       507760
ConstantFolding/llvm_folding/8                                    4257 ns         4180 ns       168245
ConstantFolding/llvm_folding/64                                  26929 ns        26840 ns        26297
ConstantFolding/llvm_folding/512                                156498 ns       156342 ns         4432
ConstantFolding/llvm_folding/1000                               289457 ns       289321 ns         2429
ConstantFolding/llvm_folding_BigO                               293.25 N        293.07 N
ConstantFolding/llvm_folding_RMS                                     5 %             5 %
Cloning/cloneOps/10                                               2220 ns         2220 ns       318436
Cloning/cloneOps/64                                              14691 ns        14690 ns        46549
Cloning/cloneOps/512                                            136508 ns       136502 ns         5122
Cloning/cloneOps/4096                                          1242184 ns      1242098 ns          565
Cloning/cloneOps/32768                                        10503352 ns     10502102 ns           66
Cloning/cloneOps/262144                                      108813716 ns    108810122 ns            5
Cloning/cloneOps/2097152                                    1521077252 ns   1521011265 ns            1
Cloning/cloneOps/10000000                                   7660330775 ns   7659616760 ns            1
Cloning/cloneOps_BigO                                           764.08 N        764.01 N
Cloning/cloneOps_RMS                                                 4 %             4 %
CreateOps/simple/10                                               1536 ns         1536 ns       460046
CreateOps/simple/64                                               9646 ns         9644 ns        72357
CreateOps/simple/512                                             78018 ns        78000 ns         9087
CreateOps/simple/4096                                           633086 ns       633009 ns         1105
CreateOps/simple/32768                                         5075687 ns      5074757 ns          141
CreateOps/simple/262144                                       39982401 ns     39974803 ns           17
CreateOps/simple/2097152                                     360437464 ns    360353166 ns            2
CreateOps/simple/10000000                                   2086232311 ns   2086075480 ns            1
CreateOps/simple_BigO                                           207.04 N        207.02 N
CreateOps/simple_RMS                                                 9 %             9 %
CreateOps/hoistedOpState/10                                       1428 ns         1428 ns       493202
CreateOps/hoistedOpState/64                                       9086 ns         9086 ns        76665
CreateOps/hoistedOpState/512                                     72140 ns        72131 ns         9599
CreateOps/hoistedOpState/4096                                   578253 ns       578227 ns         1203
CreateOps/hoistedOpState/32768                                 4646163 ns      4645889 ns          152
CreateOps/hoistedOpState/262144                               37132617 ns     37129104 ns           19
CreateOps/hoistedOpState/2097152                             294337820 ns    294303042 ns            2
CreateOps/hoistedOpState/10000000                           1415526483 ns   1415445417 ns            1
CreateOps/hoistedOpState_BigO                                   141.50 N        141.49 N
CreateOps/hoistedOpState_RMS                                         0 %             0 %
CreateOps/withInsert/10                                           1438 ns         1438 ns       457959
CreateOps/withInsert/64                                           7362 ns         7362 ns       109665
CreateOps/withInsert/512                                         48655 ns        48653 ns        13639
CreateOps/withInsert/4096                                       399650 ns       399628 ns         1781
CreateOps/withInsert/32768                                     3183901 ns      3183579 ns          220
CreateOps/withInsert/262144                                   25701754 ns     25699965 ns           28
CreateOps/withInsert/2097152                                 203392387 ns    203380240 ns            3
CreateOps/withInsert/10000000                               1091894863 ns   1091801688 ns            1
CreateOps/withInsert_BigO                                       108.67 N        108.66 N
CreateOps/withInsert_RMS                                             5 %             5 %
CreateOps/simpleRegistered/10                                     1071 ns         1070 ns       637199
CreateOps/simpleRegistered/64                                     7944 ns         7943 ns       100889
CreateOps/simpleRegistered/512                                   75907 ns        75901 ns         9140
CreateOps/simpleRegistered/4096                                 602203 ns       602160 ns         1171
CreateOps/simpleRegistered/32768                               5002004 ns      5001232 ns          100
CreateOps/simpleRegistered/262144                             38527344 ns     38525541 ns           18
CreateOps/simpleRegistered/2097152                           314170408 ns    314147116 ns            2
CreateOps/simpleRegistered/10000000                         1486016787 ns   1485892910 ns            1
CreateOps/simpleRegistered_BigO                                 148.65 N        148.64 N
CreateOps/simpleRegistered_RMS                                       0 %             0 %
CreateOps/withInsertRegistered/10                                 1529 ns         1529 ns       446050
CreateOps/withInsertRegistered/64                                 8186 ns         8186 ns        92710
CreateOps/withInsertRegistered/512                               60538 ns        60532 ns        11322
CreateOps/withInsertRegistered/4096                             469575 ns       469520 ns         1447
CreateOps/withInsertRegistered/32768                           3768329 ns      3768200 ns          183
CreateOps/withInsertRegistered/262144                         30147703 ns     30146259 ns           23
CreateOps/withInsertRegistered/2097152                       240048733 ns    240016824 ns            3
CreateOps/withInsertRegistered/10000000                     1287778683 ns   1287710159 ns            1
CreateOps/withInsertRegistered_BigO                             128.17 N        128.16 N
CreateOps/withInsertRegistered_RMS                                   5 %             5 %
CreateOps/llvm_withInsertRegistered/10                             621 ns          621 ns      1310205
CreateOps/llvm_withInsertRegistered/64                            3399 ns         3399 ns       208887
CreateOps/llvm_withInsertRegistered/512                          26754 ns        26751 ns        26019
CreateOps/llvm_withInsertRegistered/4096                        214032 ns       214024 ns         3267
CreateOps/llvm_withInsertRegistered/32768                      1740016 ns      1739836 ns          409
CreateOps/llvm_withInsertRegistered/262144                    13703840 ns     13703383 ns           51
CreateOps/llvm_withInsertRegistered/2097152                  116423890 ns    116413778 ns            7
CreateOps/llvm_withInsertRegistered/10000000                 508718534 ns    508690844 ns            1
CreateOps/llvm_withInsertRegistered_BigO                         51.07 N         51.07 N
CreateOps/llvm_withInsertRegistered_RMS                              4 %             4 %
CreateOps/simpleConstant/10                                       4259 ns         4259 ns       166102
CreateOps/simpleConstant/64                                      27216 ns        27212 ns        25925
CreateOps/simpleConstant/512                                    215266 ns       215255 ns         3259
CreateOps/simpleConstant/4096                                  1706406 ns      1706316 ns          407
CreateOps/simpleConstant/32768                                13536332 ns     13535658 ns           51
CreateOps/simpleConstant/262144                              109289148 ns    109280023 ns            6
CreateOps/simpleConstant/2097152                             872710808 ns    872649757 ns            1
CreateOps/simpleConstant/10000000                           4174540562 ns   4174213028 ns            1
CreateOps/simpleConstant_BigO                                   417.40 N        417.37 N
CreateOps/simpleConstant_RMS                                         0 %             0 %
CreateOps/simpleRegisteredConstant/10                             3584 ns         3583 ns       195947
CreateOps/simpleRegisteredConstant/64                            22858 ns        22857 ns        30869
CreateOps/simpleRegisteredConstant/512                          179839 ns       179822 ns         3900
CreateOps/simpleRegisteredConstant/4096                        1436457 ns      1436281 ns          495
CreateOps/simpleRegisteredConstant/32768                      11687039 ns     11686423 ns           61
CreateOps/simpleRegisteredConstant/262144                     93986445 ns     93981382 ns            8
CreateOps/simpleRegisteredConstant/2097152                   734716294 ns    734631856 ns            1
CreateOps/simpleRegisteredConstant/10000000                 3516873944 ns   3516440595 ns            1
CreateOps/simpleRegisteredConstant_BigO                         351.64 N        351.59 N
CreateOps/simpleRegisteredConstant_RMS                               0 %             0 %
DialectConversion/noPatterns/1                                    2114 ns         2058 ns       341049
DialectConversion/noPatterns/8                                    3206 ns         3144 ns       223480
DialectConversion/noPatterns/64                                  10719 ns        10667 ns        67748
DialectConversion/noPatterns/512                                 67336 ns        67225 ns        10457
DialectConversion/noPatterns/4096                               522999 ns       522861 ns         1337
DialectConversion/noPatterns/32768                             4157336 ns      4156524 ns          168
DialectConversion/noPatterns/262144                           34530702 ns     34527365 ns           20
DialectConversion/noPatterns/1000000                         141217638 ns    141213232 ns            5
DialectConversion/noPatterns_BigO                               140.59 N        140.59 N
DialectConversion/noPatterns_RMS                                     4 %             4 %
DialectConversion/toLLVM/1                                       26222 ns        26240 ns        26347
DialectConversion/toLLVM/8                                       52543 ns        52546 ns        13238
DialectConversion/toLLVM/64                                     270734 ns       270727 ns         2555
DialectConversion/toLLVM/512                                   1942653 ns      1942429 ns          355
DialectConversion/toLLVM/4096                                 15547873 ns     15547373 ns           45
DialectConversion/toLLVM/32768                               135306178 ns    135301871 ns            5
DialectConversion/toLLVM/262144                             1349905322 ns   1349700436 ns            1
DialectConversion/toLLVM/1000000                            6221941548 ns   6221300959 ns            1
DialectConversion/toLLVM_BigO                                  6150.91 N       6150.26 N
DialectConversion/toLLVM_RMS                                        10 %            10 %
GreedyRewriter/empty/1                                             160 ns          160 ns      4317770
GreedyRewriter/empty/8                                             310 ns          310 ns      2242470
GreedyRewriter/empty/64                                           1999 ns         1998 ns       344037
GreedyRewriter/empty/512                                         18588 ns        18584 ns        37692
GreedyRewriter/empty/4096                                       196155 ns       196116 ns         3571
GreedyRewriter/empty/32768                                     1562059 ns      1561765 ns          453
GreedyRewriter/empty/262144                                   14990618 ns     14987106 ns           47
GreedyRewriter/empty/2097152                                 161013784 ns    160978620 ns            4
GreedyRewriter/empty/10000000                               1160105739 ns   1159997571 ns            1
GreedyRewriter/empty_BigO                                       114.32 N        114.31 N
GreedyRewriter/empty_RMS                                            18 %            18 %
GreedyRewriter/withCanonicalizationPatterns/10                   40997 ns        40995 ns        16619
GreedyRewriter/withCanonicalizationPatterns/64                   43101 ns        43100 ns        16070
GreedyRewriter/withCanonicalizationPatterns/512                  61193 ns        61189 ns        11507
GreedyRewriter/withCanonicalizationPatterns/4096                229037 ns       229017 ns         2960
GreedyRewriter/withCanonicalizationPatterns/32768              1687291 ns      1686925 ns          430
GreedyRewriter/withCanonicalizationPatterns/262144            15046323 ns     15042271 ns           47
GreedyRewriter/withCanonicalizationPatterns/2097152          166651302 ns    166609746 ns            4
GreedyRewriter/withCanonicalizationPatterns/10000000        1205316641 ns   1205235646 ns            1
GreedyRewriter/withCanonicalizationPatterns_BigO                118.76 N        118.75 N
GreedyRewriter/withCanonicalizationPatterns_RMS                     17 %            17 %
GreedyRewriter/withPatterns/100                                  48255 ns        48244 ns        14615
GreedyRewriter/withPatterns/512                                  86790 ns        86771 ns         7815
GreedyRewriter/withPatterns/4096                                630096 ns       630079 ns         1172
GreedyRewriter/withPatterns/10000                              1454875 ns      1454797 ns          480
GreedyRewriter/withPatterns_BigO                                146.77 N        146.76 N
GreedyRewriter/withPatterns_RMS                                      4 %             4 %
InterfaceBench/vectorTraveralCallInterfaceMethod/10               99.8 ns         99.8 ns      7047449
InterfaceBench/vectorTraveralCallInterfaceMethod/64                591 ns          591 ns      1183230
InterfaceBench/vectorTraveralCallInterfaceMethod/512              4638 ns         4638 ns       150472
InterfaceBench/vectorTraveralCallInterfaceMethod/4096            37237 ns        37229 ns        18789
InterfaceBench/vectorTraveralCallInterfaceMethod/32768          300670 ns       300651 ns         2333
InterfaceBench/vectorTraveralCallInterfaceMethod/262144        2643092 ns      2642860 ns          263
InterfaceBench/vectorTraveralCallInterfaceMethod/2097152      20712952 ns     20708857 ns           34
InterfaceBench/vectorTraveralCallInterfaceMethod/10000000     98550537 ns     98542763 ns            7
InterfaceBench/vectorTraveralCallInterfaceMethod_BigO             9.86 N          9.86 N
InterfaceBench/vectorTraveralCallInterfaceMethod_RMS                 0 %             0 %
InterfaceBench/llvm_vectorTraveralCallInterfaceMethod/10          15.0 ns         15.0 ns     46612990
InterfaceBench/llvm_vectorTraveralCallInterfaceMethod/64          93.3 ns         93.3 ns      7470909
InterfaceBench/llvm_vectorTraveralCallInterfaceMethod/512          699 ns          699 ns       995669
InterfaceBench/llvm_vectorTraveralCallInterfaceMethod/4096        5522 ns         5521 ns       126964
InterfaceBench/llvm_vectorTraveralCallInterfaceMethod/10000      15494 ns        15493 ns        45167
InterfaceBench/llvm_vectorTraveralCallInterfaceMethod_BigO        1.52 N          1.52 N
InterfaceBench/llvm_vectorTraveralCallInterfaceMethod_RMS            8 %             8 %
LoopUnrolling/unroll/1                                            4502 ns         4410 ns       158967
LoopUnrolling/unroll/8                                           23767 ns        23618 ns        29563
LoopUnrolling/unroll/64                                         172746 ns       172507 ns         4052
LoopUnrolling/unroll/512                                       1322454 ns      1321995 ns          534
LoopUnrolling/unroll/1000                                      2545860 ns      2545325 ns          276
LoopUnrolling/unroll_BigO                                      2554.05 N       2553.43 N
LoopUnrolling/unroll_RMS                                             1 %             1 %
LoopUnrolling/llvm_unroll/1                                      36144 ns        36123 ns        19441
LoopUnrolling/llvm_unroll/8                                     255243 ns       255368 ns         2735
LoopUnrolling/llvm_unroll/64                                   2700037 ns      2700070 ns          259
LoopUnrolling/llvm_unroll/512                                 73407469 ns     73402333 ns           10
LoopUnrolling/llvm_unroll/1000                               260718746 ns    260696177 ns            3
LoopUnrolling/llvm_unroll_BigO                               235708.03 N     235688.13 N
LoopUnrolling/llvm_unroll_RMS                                       36 %            36 %
ParserPrinter/parseTextualIR/10                                  61549 ns        61546 ns        11338
ParserPrinter/parseTextualIR/64                                 179556 ns       179544 ns         3894
ParserPrinter/parseTextualIR/512                               1220245 ns      1220158 ns          573
ParserPrinter/parseTextualIR/4096                              9856960 ns      9855832 ns           71
ParserPrinter/parseTextualIR/32768                            82664124 ns     82658142 ns            8
ParserPrinter/parseTextualIR/262144                          995924502 ns    995851929 ns            1
ParserPrinter/parseTextualIR/1000000                        4324001717 ns   4323683749 ns            1
ParserPrinter/parseTextualIR_BigO                              4288.45 N       4288.13 N
ParserPrinter/parseTextualIR_RMS                                     7 %             7 %
ParserPrinter/parseBytecodeIR/10                                 47988 ns        47984 ns        14588
ParserPrinter/parseBytecodeIR/64                                100307 ns       100281 ns         6952
ParserPrinter/parseBytecodeIR/512                               573133 ns       573025 ns         1218
ParserPrinter/parseBytecodeIR/4096                             4586582 ns      4585625 ns          153
ParserPrinter/parseBytecodeIR/32768                           37373868 ns     37371286 ns           19
ParserPrinter/parseBytecodeIR/262144                         420982484 ns    420959945 ns            2
ParserPrinter/parseBytecodeIR/1000000                       2034472944 ns   2034323491 ns            1
ParserPrinter/parseBytecodeIR_BigO                             2006.03 N       2005.89 N
ParserPrinter/parseBytecodeIR_RMS                                   12 %            12 %
ParserPrinter/printTextualIR/10                                  14484 ns        14482 ns        48128
ParserPrinter/printTextualIR/64                                  62662 ns        62660 ns        11078
ParserPrinter/printTextualIR/512                                463614 ns       463600 ns         1507
ParserPrinter/printTextualIR/4096                              3802635 ns      3802384 ns          184
ParserPrinter/printTextualIR/32768                            31159720 ns     31152274 ns           22
ParserPrinter/printTextualIR/262144                          298570499 ns    298498153 ns            2
ParserPrinter/printTextualIR/1000000                        1246963499 ns   1246833554 ns            1
ParserPrinter/printTextualIR_BigO                              1239.72 N       1239.58 N
ParserPrinter/printTextualIR_RMS                                     5 %             5 %
ParserPrinter/printBytecodeIR/10                                 12809 ns        12805 ns        53762
ParserPrinter/printBytecodeIR/64                                 49801 ns        49790 ns        13943
ParserPrinter/printBytecodeIR/512                               408265 ns       408185 ns         1735
ParserPrinter/printBytecodeIR/4096                             3312854 ns      3312253 ns          209
ParserPrinter/printBytecodeIR/32768                           28967742 ns     28958923 ns           23
ParserPrinter/printBytecodeIR/262144                         299572774 ns    299555963 ns            2
ParserPrinter/printBytecodeIR/1000000                       1386730152 ns   1386637183 ns            1
ParserPrinter/printBytecodeIR_BigO                             1370.55 N       1370.45 N
ParserPrinter/printBytecodeIR_RMS                                   10 %            10 %
IRWalk/blockTraveral/10                                           17.5 ns         17.5 ns     39881680
IRWalk/blockTraveral/64                                            110 ns          110 ns      6379697
IRWalk/blockTraveral/512                                          1596 ns         1595 ns       439735
IRWalk/blockTraveral/4096                                        10195 ns        10194 ns        68860
IRWalk/blockTraveral/32768                                       81073 ns        81055 ns         8650
IRWalk/blockTraveral/262144                                    1841230 ns      1840690 ns          381
IRWalk/blockTraveral/2097152                                  16323138 ns     16319776 ns           43
IRWalk/blockTraveral/10000000                                 80215734 ns     80210678 ns            9
IRWalk/blockTraveral_BigO                                         8.01 N          8.01 N
IRWalk/blockTraveral_RMS                                             2 %             2 %
IRWalk/vectorTraveral/10                                          4.06 ns         4.06 ns    171981038
IRWalk/vectorTraveral/64                                          26.4 ns         26.4 ns     26457066
IRWalk/vectorTraveral/512                                          232 ns          232 ns      3020721
IRWalk/vectorTraveral/4096                                        1849 ns         1849 ns       379201
IRWalk/vectorTraveral/32768                                      14762 ns        14759 ns        47414
IRWalk/vectorTraveral/262144                                    114403 ns       114397 ns         6116
IRWalk/vectorTraveral/2097152                                  1028901 ns      1028851 ns          685
IRWalk/vectorTraveral/10000000                                 6133232 ns      6132622 ns          114
IRWalk/vectorTraveral_BigO                                        0.61 N          0.61 N
IRWalk/vectorTraveral_RMS                                           10 %            10 %
IRWalk/simpleWalk/10                                              62.6 ns         62.6 ns     11116850
IRWalk/simpleWalk/64                                               350 ns          350 ns      1999507
IRWalk/simpleWalk/512                                             2729 ns         2729 ns       256518
IRWalk/simpleWalk/4096                                           21833 ns        21831 ns        32038
IRWalk/simpleWalk/32768                                         176236 ns       176223 ns         3985
IRWalk/simpleWalk/262144                                       1742919 ns      1742834 ns          404
IRWalk/simpleWalk/2097152                                     14132705 ns     14131555 ns           49
IRWalk/simpleWalk/10000000                                    67449778 ns     67445037 ns           10
IRWalk/simpleWalk_BigO                                            6.74 N          6.74 N
IRWalk/simpleWalk_RMS                                                0 %             0 %
IRWalk/filteredOps/10                                             17.1 ns         17.1 ns     40873782
IRWalk/filteredOps/64                                              113 ns          113 ns      6174719
IRWalk/filteredOps/512                                            1603 ns         1603 ns       431125
IRWalk/filteredOps/4096                                           9621 ns         9620 ns        72549
IRWalk/filteredOps/32768                                         76871 ns        76866 ns         9082
IRWalk/filteredOps/262144                                      1558707 ns      1558516 ns          452
IRWalk/filteredOps/2097152                                    13222558 ns     13221336 ns           53
IRWalk/filteredOps/10000000                                   62731864 ns     62725928 ns           11
IRWalk/filteredOps_BigO                                           6.27 N          6.27 N
IRWalk/filteredOps_RMS                                               1 %             1 %
IRWalk/nestedRegion/10                                            84.7 ns         84.7 ns      8264952
IRWalk/nestedRegion/16                                             125 ns          125 ns      5587643
IRWalk/nestedRegion/64                                             454 ns          454 ns      1531392
IRWalk/nestedRegion/256                                           1828 ns         1828 ns       384758
IRWalk/nestedRegion/1024                                          7129 ns         7129 ns        91981
IRWalk/nestedRegion/4096                                         30161 ns        30159 ns        23193
IRWalk/nestedRegion/8000                                         57663 ns        57658 ns        12012
IRWalk/nestedRegion_BigO                                          7.24 N          7.24 N
IRWalk/nestedRegion_RMS                                              2 %             2 %
IRWalk/llvm_blockTraversal/10                                     7.48 ns         7.48 ns     93670794
IRWalk/llvm_blockTraversal/64                                     69.4 ns         69.4 ns     10072389
IRWalk/llvm_blockTraversal/512                                    1476 ns         1476 ns       469258
IRWalk/llvm_blockTraversal/4096                                   9975 ns         9975 ns        70587
IRWalk/llvm_blockTraversal/32768                                 80465 ns        80459 ns         8673
IRWalk/llvm_blockTraversal/262144                              1803648 ns      1803592 ns          390
IRWalk/llvm_blockTraversal/2097152                            16225756 ns     16225254 ns           43
IRWalk/llvm_blockTraversal/10000000                           78978276 ns     78975659 ns            9
IRWalk/llvm_blockTraversal_BigO                                   7.89 N          7.89 N
IRWalk/llvm_blockTraversal_RMS                                       1 %             1 %
IRWalk/vectorTraveralOpCastFail/10                                7.36 ns         7.36 ns     95088104
IRWalk/vectorTraveralOpCastFail/64                                45.7 ns         45.7 ns     15317168
IRWalk/vectorTraveralOpCastFail/512                                476 ns          476 ns      1450136
IRWalk/vectorTraveralOpCastFail/4096                              3790 ns         3790 ns       184584
IRWalk/vectorTraveralOpCastFail/32768                            39812 ns        39809 ns        17634
IRWalk/vectorTraveralOpCastFail/262144                          975434 ns       975405 ns          720
IRWalk/vectorTraveralOpCastFail/2097152                        8580938 ns      8580456 ns           81
IRWalk/vectorTraveralOpCastFail/10000000                      40768999 ns     40764746 ns           17
IRWalk/vectorTraveralOpCastFail_BigO                              4.08 N          4.08 N
IRWalk/vectorTraveralOpCastFail_RMS                                  1 %             1 %
IRWalk/vectorTraveralOpCastSuccess/10                             7.39 ns         7.39 ns     93692158
IRWalk/vectorTraveralOpCastSuccess/64                             45.7 ns         45.7 ns     15323919
IRWalk/vectorTraveralOpCastSuccess/512                             523 ns          523 ns      1294067
IRWalk/vectorTraveralOpCastSuccess/4096                           3785 ns         3785 ns       186404
IRWalk/vectorTraveralOpCastSuccess/32768                         39972 ns        39970 ns        17502
IRWalk/vectorTraveralOpCastSuccess/262144                       977336 ns       977306 ns          720
IRWalk/vectorTraveralOpCastSuccess/2097152                     8450122 ns      8447973 ns           81
IRWalk/vectorTraveralOpCastSuccess/10000000                   40680953 ns     40678330 ns           17
IRWalk/vectorTraveralOpCastSuccess_BigO                           4.07 N          4.07 N
IRWalk/vectorTraveralOpCastSuccess_RMS                               1 %             1 %
IRWalk/vectorTraveralWithoutInterfaceCastFail/10                  39.2 ns         39.2 ns     17857367
IRWalk/vectorTraveralWithoutInterfaceCastFail/64                   254 ns          254 ns      2761824
IRWalk/vectorTraveralWithoutInterfaceCastFail/512                 2132 ns         2131 ns       328156
IRWalk/vectorTraveralWithoutInterfaceCastFail/4096               16273 ns        16272 ns        42906
IRWalk/vectorTraveralWithoutInterfaceCastFail/32768             131320 ns       131283 ns         5341
IRWalk/vectorTraveralWithoutInterfaceCastFail/262144           1665353 ns      1664450 ns          422
IRWalk/vectorTraveralWithoutInterfaceCastFail/2097152         13372446 ns     13369410 ns           52
IRWalk/vectorTraveralWithoutInterfaceCastFail/10000000        63311417 ns     63307361 ns           11
IRWalk/vectorTraveralWithoutInterfaceCastFail_BigO                6.33 N          6.33 N
IRWalk/vectorTraveralWithoutInterfaceCastFail_RMS                    0 %             0 %
IRWalk/vectorTraveralWithInterfaceCastFail/10                     47.3 ns         47.3 ns     14787937
IRWalk/vectorTraveralWithInterfaceCastFail/64                      308 ns          308 ns      2278677
IRWalk/vectorTraveralWithInterfaceCastFail/512                    2458 ns         2458 ns       285080
IRWalk/vectorTraveralWithInterfaceCastFail/4096                  19516 ns        19515 ns        35921
IRWalk/vectorTraveralWithInterfaceCastFail/32768                156901 ns       156861 ns         4470
IRWalk/vectorTraveralWithInterfaceCastFail/262144              2045579 ns      2045320 ns          338
IRWalk/vectorTraveralWithInterfaceCastFail/2097152            16461268 ns     16455022 ns           43
IRWalk/vectorTraveralWithInterfaceCastFail/10000000           78230340 ns     78221750 ns            9
IRWalk/vectorTraveralWithInterfaceCastFail_BigO                   7.82 N          7.82 N
IRWalk/vectorTraveralWithInterfaceCastFail_RMS                       0 %             0 %
IRWalk/vectorTraveralWithInterfaceCastSuccess/10                  85.7 ns         85.7 ns      8214303
IRWalk/vectorTraveralWithInterfaceCastSuccess/64                   489 ns          489 ns      1431791
IRWalk/vectorTraveralWithInterfaceCastSuccess/512                 3933 ns         3933 ns       179271
IRWalk/vectorTraveralWithInterfaceCastSuccess/4096               30951 ns        30949 ns        22630
IRWalk/vectorTraveralWithInterfaceCastSuccess/32768             248877 ns       248870 ns         2805
IRWalk/vectorTraveralWithInterfaceCastSuccess/262144           2338713 ns      2337678 ns          300
IRWalk/vectorTraveralWithInterfaceCastSuccess/2097152         18149006 ns     18146236 ns           38
IRWalk/vectorTraveralWithInterfaceCastSuccess/10000000        88021965 ns     88009654 ns            8
IRWalk/vectorTraveralWithInterfaceCastSuccess_BigO                8.80 N          8.79 N
IRWalk/vectorTraveralWithInterfaceCastSuccess_RMS                    1 %             1 %
IRWalk/vectorTraveralOpTraitFail/10                               83.6 ns         83.6 ns      8404678
IRWalk/vectorTraveralOpTraitFail/64                                543 ns          543 ns      1292619
IRWalk/vectorTraveralOpTraitFail/512                              4284 ns         4284 ns       163011
IRWalk/vectorTraveralOpTraitFail/4096                            34317 ns        34315 ns        20364
IRWalk/vectorTraveralOpTraitFail/32768                          278203 ns       278124 ns         2511
IRWalk/vectorTraveralOpTraitFail/262144                        2520282 ns      2519074 ns          281
IRWalk/vectorTraveralOpTraitFail/2097152                      19620514 ns     19610789 ns           36
IRWalk/vectorTraveralOpTraitFail/10000000                     93801856 ns     93792653 ns            7
IRWalk/vectorTraveralOpTraitFail_BigO                             9.38 N          9.38 N
IRWalk/vectorTraveralOpTraitFail_RMS                                 0 %             0 %
IRWalk/vectorTraveralOpTraitSuccess/10                             117 ns          117 ns      5977891
IRWalk/vectorTraveralOpTraitSuccess/64                             749 ns          749 ns       935541
IRWalk/vectorTraveralOpTraitSuccess/512                           5927 ns         5927 ns       118265
IRWalk/vectorTraveralOpTraitSuccess/4096                         47596 ns        47592 ns        14704
IRWalk/vectorTraveralOpTraitSuccess/32768                       383662 ns       383566 ns         1826
IRWalk/vectorTraveralOpTraitSuccess/262144                     3441962 ns      3440234 ns          202
IRWalk/vectorTraveralOpTraitSuccess/2097152                   27770850 ns     27769434 ns           25
IRWalk/vectorTraveralOpTraitSuccess/10000000                 133148747 ns    133141848 ns            5
IRWalk/vectorTraveralOpTraitSuccess_BigO                         13.31 N         13.31 N
IRWalk/vectorTraveralOpTraitSuccess_RMS                              0 %             0 %
SimpleConstantFolding/folding/1                                  10112 ns        10070 ns        70189
SimpleConstantFolding/folding/8                                  81970 ns        81663 ns         8557
SimpleConstantFolding/folding/64                                640847 ns       638256 ns         1074
SimpleConstantFolding/folding/512                              5222675 ns      5202763 ns          135
SimpleConstantFolding/folding/4096                            42047737 ns     41843492 ns           17
SimpleConstantFolding/folding/10000                          100531426 ns    100120405 ns            7
SimpleConstantFolding/folding_BigO                            10083.92 N      10041.57 N
SimpleConstantFolding/folding_RMS                                    1 %             1 %
Dynamism/staticCall/10                                            6.26 ns         6.26 ns    112124571
Dynamism/staticCall/64                                            40.1 ns         40.1 ns     17466028
Dynamism/staticCall/512                                            322 ns          321 ns      2175745
Dynamism/staticCall/4096                                          2557 ns         2557 ns       273963
Dynamism/staticCall/32768                                        20471 ns        20470 ns        34165
Dynamism/staticCall/262144                                      163609 ns       163594 ns         4269
Dynamism/staticCall/2097152                                    1312188 ns      1312113 ns          533
Dynamism/staticCall/10000000                                   6262595 ns      6262255 ns          112
Dynamism/staticCall_BigO                                          0.63 N          0.63 N
Dynamism/staticCall_RMS                                              0 %             0 %
Dynamism/regularCall/10                                           6.26 ns         6.26 ns    111764395
Dynamism/regularCall/64                                           40.0 ns         40.0 ns     17470002
Dynamism/regularCall/512                                           321 ns          321 ns      2179870
Dynamism/regularCall/4096                                         2561 ns         2560 ns       273400
Dynamism/regularCall/32768                                       20455 ns        20454 ns        34053
Dynamism/regularCall/262144                                     163783 ns       163770 ns         4263
Dynamism/regularCall/2097152                                   1310995 ns      1310931 ns          535
Dynamism/regularCall/10000000                                  6253406 ns      6253148 ns          112
Dynamism/regularCall_BigO                                         0.63 N          0.63 N
Dynamism/regularCall_RMS                                             0 %             0 %
Dynamism/monomorphicCall/10                                       6.25 ns         6.25 ns    111854530
Dynamism/monomorphicCall/64                                       40.0 ns         40.0 ns     17499177
Dynamism/monomorphicCall/512                                       321 ns          321 ns      2183542
Dynamism/monomorphicCall/4096                                     2559 ns         2559 ns       273615
Dynamism/monomorphicCall/32768                                   20525 ns        20524 ns        34149
Dynamism/monomorphicCall/262144                                 163929 ns       163922 ns         4266
Dynamism/monomorphicCall/2097152                               1311524 ns      1311421 ns          533
Dynamism/monomorphicCall/10000000                              6250365 ns      6250010 ns          112
Dynamism/monomorphicCall_BigO                                     0.63 N          0.63 N
Dynamism/monomorphicCall_RMS                                         0 %             0 %
Dynamism/polymorphicCall/10                                       6.25 ns         6.24 ns    112007004
Dynamism/polymorphicCall/64                                       40.0 ns         40.0 ns     17488901
Dynamism/polymorphicCall/512                                       321 ns          321 ns      2180507
Dynamism/polymorphicCall/4096                                     2561 ns         2561 ns       273383
Dynamism/polymorphicCall/32768                                   20489 ns        20488 ns        34229
Dynamism/polymorphicCall/262144                                 163874 ns       163866 ns         4275
Dynamism/polymorphicCall/2097152                               1310244 ns      1310168 ns          533
Dynamism/polymorphicCall/10000000                              6253030 ns      6252564 ns          112
Dynamism/polymorphicCall_BigO                                     0.63 N          0.63 N
Dynamism/polymorphicCall_RMS                                         0 %             0 %
Dynamism/monomorphicPointerCall/10                                20.0 ns         20.0 ns     34899795
Dynamism/monomorphicPointerCall/64                                 135 ns          135 ns      5160311
Dynamism/monomorphicPointerCall/512                                976 ns          976 ns       718018
Dynamism/monomorphicPointerCall/4096                              7695 ns         7694 ns        90986
Dynamism/monomorphicPointerCall/32768                            61474 ns        61471 ns        11403
Dynamism/monomorphicPointerCall/262144                          491416 ns       491398 ns         1424
Dynamism/monomorphicPointerCall/2097152                        3934594 ns      3934387 ns          178
Dynamism/monomorphicPointerCall/10000000                      18742210 ns     18740882 ns           37
Dynamism/monomorphicPointerCall_BigO                              1.87 N          1.87 N
Dynamism/monomorphicPointerCall_RMS                                  0 %             0 %
Dynamism/polymorphicPointerCall/10                                22.0 ns         22.0 ns     31675166
Dynamism/polymorphicPointerCall/64                                 137 ns          137 ns      5091049
Dynamism/polymorphicPointerCall/512                                976 ns          976 ns       717356
Dynamism/polymorphicPointerCall/4096                              7691 ns         7691 ns        90940
Dynamism/polymorphicPointerCall/32768                            61518 ns        61514 ns        11392
Dynamism/polymorphicPointerCall/262144                          492450 ns       492430 ns         1422
Dynamism/polymorphicPointerCall/2097152                        3935077 ns      3934911 ns          178
Dynamism/polymorphicPointerCall/10000000                      18776569 ns     18775592 ns           37
Dynamism/polymorphicPointerCall_BigO                              1.88 N          1.88 N
Dynamism/polymorphicPointerCall_RMS                                  0 %             0 %
Dynamism/hasTrait/10                                               120 ns          120 ns      5842524
Dynamism/hasTrait/64                                               778 ns          778 ns       903406
Dynamism/hasTrait/512                                             6150 ns         6150 ns       113278
Dynamism/hasTrait/4096                                           49506 ns        49501 ns        14140
Dynamism/hasTrait/32768                                         397000 ns       396948 ns         1763
Dynamism/hasTrait/262144                                       3490670 ns      3490570 ns          200
Dynamism/hasTrait/2097152                                     28636495 ns     28634484 ns           25
Dynamism/hasTrait/10000000                                   140535854 ns    140523393 ns            5
Dynamism/hasTrait_BigO                                           14.04 N         14.04 N
Dynamism/hasTrait_RMS                                                1 %             1 %
    \end{minted}
    \caption{Results for the ``How Slow is MLIR?'' micro-benchmarks.}
    \label{listing:how-slow-is-mlir-microbenchmark-results}
\end{code}

\chapter{xDSL benchmark results}
\label{chap:xdsl-benchmark-results}

\section{Pipeline phase micro-benchmark results}
% Commands to run
% Table of results

\section{Micro-benchmark results}

% Hook
The following section describes the procedure and provides the raw results from the xDSL micro-benchmarks derived from ``How Slow is MLIR's''.
% Argument
% Link

\vspace{2em}

\begin{code}
    \begin{minted}{bash}
        git clone https://github.com/xdslproject/xdsl
        make venv
        python3 benchmarks/microbenchmarks all timeit
    \end{minted}
    \caption{Bash commands to download, setup the environment for, and run the benchmarks for xDSL derived from from ``How Slow is MLIR''.}
    \label{listing:bash-xdsl-ubench-run}
\end{code}

\vspace{2em}

\begin{code}
    \begin{minted}[fontsize=\scriptsize]{text}
Test IRTraversal.iterate_ops ran in: 0.000326 ± 5.92e-06s // ÷ 32768
Test IRTraversal.iterate_block_ops ran in: 0.00675 ± 4.5e-05s // ÷ 32768
Test IRTraversal.walk_block_ops ran in: 0.0172 ± 9.31e-05s // ÷ 32768
Test Extensibility.interface_check_trait ran in: 4.72e-08 ± 4.27e-08s
Test Extensibility.interface_check ran in: 4.38e-08 ± 3.09e-08s
Test Extensibility.trait_check ran in: 9.24e-07 ± 5.13e-07s
Test Extensibility.trait_check_optimised ran in: 2.66e-07 ± 3.4e-08s
Test Extensibility.trait_check_single ran in: 6.72e-07 ± 5.11e-07s
Test Extensibility.trait_check_neg ran in: 2.04e-06 ± 7.99e-07s
Test OpCreation.operation_create ran in: 3.77e-06 ± 9.16e-07s
Test OpCreation.operation_build ran in: 1.27e-05 ± 1.81e-06s
Test OpCreation.operation_create_optimised ran in: 4.77e-07 ± 3.85e-07s
Test OpCreation.operation_clone ran in: 0.000962 ± 1.15e-05s
Test OpCreation.operation_clone_single ran in: 6.96e-06 ± 1.55e-06s
Test OpCreation.operation_constant_init ran in: 3.34e-05 ± 3.68e-06s
Test OpCreation.operation_constant_create ran in: 1.83e-06 ± 7.74e-07s
    \end{minted}
    \caption{Results for the xDSL micro-benchmarks derived from ``How Slow is MLIR?'', for CPython version 3.10.17.}
    \label{listing:how-slow-is-mlir-xdsl-microbenchmark-results}
\end{code}

\chapter{Bytecode profiles}
\label{chap:bytecode-profiles}

\begin{code}
    \begin{minted}[fontsize=\footnotesize]{text}
//// Trace of `Extensibility.time_trait_check` :

// == microbenchmarks:176 `time_trait_check` ===
// >>> assert Extensibility.OP_WITH_REGION.has_trait(Extensibility.TRAIT_4)
204         0   LOAD_GLOBAL          0   (Extensibility)                                            // 17   ns
            2   LOAD_ATTR            1   (OP_WITH_REGION)                                           // 18   ns
            4   LOAD_METHOD          2   (has_trait)                                                // 18   ns
            6   LOAD_GLOBAL          0   (Extensibility)                                            // 17   ns
            8   LOAD_ATTR            3   (TRAIT_4)                                                  // 18   ns
            10  CALL_METHOD          1   ()                                                         // 54   ns

    // =========== core:1192 `has_trait` ===========
    // >>> from xdsl.dialects.builtin import UnregisteredOp
    1204         0   LOAD_CONST           1   (0)                                                   // 26   ns
                2   LOAD_CONST           2   (('UnregisteredOp',))                                  // 25   ns
                4   IMPORT_NAME          0   (xdsl.dialects.builtin)                                // 72   ns
                6   IMPORT_FROM          1   (UnregisteredOp)                                       // 28   ns
                8   STORE_FAST           3   (UnregisteredOp)                                       // 23   ns
                10  POP_TOP                  ()                                                     // 21   ns
    // >>> if issubclass(cls, UnregisteredOp):
    1206         12  LOAD_GLOBAL          2   (issubclass)                                          // 21   ns
                14  LOAD_FAST            0   (cls)                                                  // 21   ns
                16  LOAD_FAST            3   (UnregisteredOp)                                       // 19   ns
                18  CALL_FUNCTION        2   ()                                                     // 46   ns

        // ======== abc:121 `__subclasscheck__` ========
        // >>> return _abc_subclasscheck(cls, subclass)
        123         0   LOAD_GLOBAL          0   (_abc_subclasscheck)                               // 19   ns
                    2   LOAD_FAST            0   (cls)                                              // 18   ns
                    4   LOAD_FAST            1   (subclass)                                         // 17   ns
                    6   CALL_FUNCTION        2   ()                                                 // 26   ns
                    8   RETURN_VALUE             ()                                                 // 44   ns
        // =============================================

                20  POP_JUMP_IF_FALSE    13  (to 26)                                                // 22   ns
    // >>> return cls.get_trait(trait) is not None
    1209     >>  26  LOAD_FAST            0   (cls)                                                 // 21   ns
                28  LOAD_METHOD          3   (get_trait)                                            // 23   ns
                30  LOAD_FAST            1   (trait)                                                // 21   ns
                32  CALL_METHOD          1   ()                                                     // 52   ns

        // =========== core:1211 `get_trait` ===========
        // >>> if isinstance(trait, type):
        1216         0   LOAD_GLOBAL          0   (isinstance)                                      // 24   ns
                    2   LOAD_FAST            1   (trait)                                            // 23   ns
                    4   LOAD_GLOBAL          1   (type)                                             // 25   ns
                    6   CALL_FUNCTION        2   ()                                                 // 25   ns
                    8   POP_JUMP_IF_FALSE    27  (to 54)                                            // 23   ns
        // >>> for t in cls.traits:
        1217         10  LOAD_FAST            0   (cls)                                             // 22   ns
                    12  LOAD_ATTR            2   (traits)                                           // 24   ns
                    14  GET_ITER                 ()                                                 // 43   ns

            // ============ core:758 `__iter__` ============
            // >>> return iter(self.traits)
            759         0   LOAD_GLOBAL          0   (iter)                                         // 17   ns
                        2   LOAD_FAST            0   (self)                                         // 16   ns
                        4   LOAD_ATTR            1   (traits)                                       // 38   ns

                // ============= core:747 `traits` =============
                // >>> if callable(self._traits):
                750         0   LOAD_GLOBAL          0   (callable)                                 // 17   ns
                            2   LOAD_FAST            0   (self)                                     // 15   ns
                            4   LOAD_ATTR            1   (_traits)                                  // 17   ns
                            6   CALL_FUNCTION        1   ()                                         // 18   ns
                            8   POP_JUMP_IF_FALSE    12  (to 24)                                    // 17   ns
                // >>> return self._traits
                752     >>  24  LOAD_FAST            0   (self)                                     // 16   ns
                            26  LOAD_ATTR            1   (_traits)                                  // 16   ns
                            28  RETURN_VALUE             ()                                         // 38   ns
                // =============================================

                        6   CALL_FUNCTION        1   ()                                             // 22   ns
                        8   RETURN_VALUE             ()                                             // 42   ns
            // =============================================

                >>  16  FOR_ITER             16  (to 50)                                            // 25   ns
                    18  STORE_FAST           2   (t)                                                // 22   ns
        // >>> if isinstance(t, cast(type[OpTraitInvT], trait)):
        1218         20  LOAD_GLOBAL          0   (isinstance)                                      // 21   ns
                    22  LOAD_FAST            2   (t)                                                // 21   ns
                    24  LOAD_GLOBAL          3   (cast)                                             // 21   ns
                    26  LOAD_GLOBAL          1   (type)                                             // 19   ns
                    28  LOAD_GLOBAL          4   (OpTraitInvT)                                      // 18   ns
                    30  BINARY_SUBSCR            ()                                                 // 22   ns
                    32  LOAD_FAST            1   (trait)                                            // 19   ns
                    34  CALL_FUNCTION        2   ()                                                 // 41   ns

            // ============ typing:1737 `cast` =============
            // >>> return val
            1745         0   LOAD_FAST            1   (val)                                         // 18   ns
                        2   RETURN_VALUE             ()                                             // 43   ns
            // =============================================

                    36  CALL_FUNCTION        2   ()                                                 // 23   ns
                    38  POP_JUMP_IF_FALSE    24  (to 48)                                            // 20   ns
        1218     >>  48  JUMP_ABSOLUTE        8   (to 16)                                           // 20   ns
        // >>> for t in cls.traits:
                >>  16  FOR_ITER             16  (to 50)                                            // 19   ns
                    18  STORE_FAST           2   (t)                                                // 19   ns
        // >>> if isinstance(t, cast(type[OpTraitInvT], trait)):
        1218         20  LOAD_GLOBAL          0   (isinstance)                                      // 19   ns
                    22  LOAD_FAST            2   (t)                                                // 19   ns
                    24  LOAD_GLOBAL          3   (cast)                                             // 19   ns
                    26  LOAD_GLOBAL          1   (type)                                             // 18   ns
                    28  LOAD_GLOBAL          4   (OpTraitInvT)                                      // 18   ns
                    30  BINARY_SUBSCR            ()                                                 // 21   ns
                    32  LOAD_FAST            1   (trait)                                            // 19   ns
                    34  CALL_FUNCTION        2   ()                                                 // 40   ns

            // ============ typing:1737 `cast` =============
            // >>> return val
            1745         0   LOAD_FAST            1   (val)                                         // 18   ns
                        2   RETURN_VALUE             ()                                             // 43   ns
            // =============================================

                    36  CALL_FUNCTION        2   ()                                                 // 21   ns
                    38  POP_JUMP_IF_FALSE    24  (to 48)                                            // 20   ns
        // >>> return t
        1219         40  LOAD_FAST            2   (t)                                               // 19   ns
                    42  ROT_TWO                  ()                                                 // 19   ns
                    44  POP_TOP                  ()                                                 // 20   ns
                    46  RETURN_VALUE             ()                                                 // 47   ns
        // =============================================

                34  LOAD_CONST           3   (None)                                                 // 21   ns
                36  IS_OP                1   ()                                                     // 21   ns
                38  RETURN_VALUE             ()                                                     // 47   ns
    // =============================================

            12  POP_JUMP_IF_TRUE     9   (to 18)                                                    // 16   ns
        >>  18  LOAD_CONST           1   (None)                                                     // 16   ns
            20  RETURN_VALUE             ()                                                         // 39   ns
// =============================================
    \end{minted}
    \caption{Bytecode profile trace of the original implementation of \mintinline{python}{has_trait}.}
    \label{listing:bytecode-profiles-hastrait-original}
\end{code}

\vspace{2em}

\begin{code}
    \begin{minted}[fontsize=\footnotesize]{text}
//// Trace of `Extensibility.time_trait_check_optimised` :

// == microbenchmarks:206 `time_trait_check_optimised` ==
// >>> has_trait = False
208         0   LOAD_CONST           1   (False)                                                    // 14   ns
            2   STORE_FAST           1   (has_trait)                                                // 14   ns
// >>> for t in Extensibility.OP_WITH_REGION.traits._traits:
209         4   LOAD_GLOBAL          0   (Extensibility)                                            // 13   ns
            6   LOAD_ATTR            1   (OP_WITH_REGION)                                           // 13   ns
            8   LOAD_ATTR            2   (traits)                                                   // 13   ns
            10  LOAD_ATTR            3   (_traits)                                                  // 13   ns
            12  GET_ITER                 ()                                                         // 13   ns
        >>  14  FOR_ITER             12  (to 40)                                                    // 13   ns
            16  STORE_FAST           2   (t)                                                        // 11   ns
// >>> if isinstance(t, Extensibility.TRAIT_4):
210         18  LOAD_GLOBAL          4   (isinstance)                                               // 11   ns
            20  LOAD_FAST            2   (t)                                                        // 11   ns
            22  LOAD_GLOBAL          0   (Extensibility)                                            // 11   ns
            24  LOAD_ATTR            5   (TRAIT_4)                                                  // 11   ns
            26  CALL_FUNCTION        2   ()                                                         // 12   ns
            28  POP_JUMP_IF_FALSE    19  (to 38)                                                    // 11   ns
210     >>  38  JUMP_ABSOLUTE        7   (to 14)                                                    // 11   ns
// >>> for t in Extensibility.OP_WITH_REGION.traits._traits:
        >>  14  FOR_ITER             12  (to 40)                                                    // 11   ns
            16  STORE_FAST           2   (t)                                                        // 11   ns
// >>> if isinstance(t, Extensibility.TRAIT_4):
210         18  LOAD_GLOBAL          4   (isinstance)                                               // 11   ns
            20  LOAD_FAST            2   (t)                                                        // 10   ns
            22  LOAD_GLOBAL          0   (Extensibility)                                            // 11   ns
            24  LOAD_ATTR            5   (TRAIT_4)                                                  // 11   ns
            26  CALL_FUNCTION        2   ()                                                         // 11   ns
            28  POP_JUMP_IF_FALSE    19  (to 38)                                                    // 11   ns
// >>> has_trait = True
211         30  LOAD_CONST           2   (True)                                                     // 11   ns
            32  STORE_FAST           1   (has_trait)                                                // 11   ns
// >>> break
212         34  POP_TOP                  ()                                                         // 11   ns
            36  JUMP_FORWARD         1   (to 40)                                                    // 11   ns
// >>> assert has_trait
213     >>  40  LOAD_FAST            1   (has_trait)                                                // 11   ns
            42  POP_JUMP_IF_TRUE     24  (to 48)                                                    // 11   ns
        >>  48  LOAD_CONST           3   (None)                                                     // 11   ns
            50  RETURN_VALUE             ()                                                         // 23   ns
// =============================================
    \end{minted}
    \caption{Bytecode profile trace of the optimised implementation of \mintinline{python}{has_trait}.}
    \label{listing:bytecode-profiles-hastrait-optimised}
\end{code}

\chapter{Disassembly of dynamic dispatch experiments}
\label{chap:impact-disassembly}


\vspace{2em}
\begin{code}
    \begin{minted}[fontsize=\scriptsize]{text}
main:
 stp	x29, x30, [sp, #-32]!
 str	x19, [sp, #16]
 mov	x29, sp
 ldr	x0, [x1, #8]
 mov	x19, x1
 mov	x1, xzr
 mov	w2, #0xa                   	// #10
 bl	0 <strtol>
    R_AARCH64_CALL26 strtol
 ldr	x8, [x19, #16]
 mov	x19, x0
 mov	x1, xzr
 mov	w2, #0xa                   	// #10
 mov	x0, x8
 bl	0 <strtol>
    R_AARCH64_CALL26 strtol
// ===== Start of function invocation ===== //
 sub	w0, w19, w0
// ====== End of function invocation ====== //
 ldr	x19, [sp, #16]
 ldp	x29, x30, [sp], #32
 ret
    \end{minted}
    \caption{Disassembly of inlined function invocation (Listing \ref{listing:impact-dispatch-definition} \circledbase{pairedTwoDarkBlue}{\scriptsize{1}}).}
    \label{listing:impact-dispatch-inlined-disassembly}
\end{code}


\vspace{2em}
\begin{code}
    \begin{minted}[fontsize=\scriptsize]{text}
main:
 stp	x29, x30, [sp, #-32]!
 str	x19, [sp, #16]
 mov	x29, sp
 ldr	x0, [x1, #8]
 mov	x19, x1
 mov	x1, xzr
 mov	w2, #0xa                   	// #10
 bl	0 <strtol>
    R_AARCH64_CALL26 strtol
 ldr	x8, [x19, #16]
 mov	x19, x0
 mov	x1, xzr
 mov	w2, #0xa                   	// #10
 mov	x0, x8
 bl	0 <strtol>
    R_AARCH64_CALL26 strtol
// ===== Start of function invocation ===== //
 mov	x2, x0
 add	x0, x29, #0x1f
 mov	w1, w19
 bl	0 <main>
    R_AARCH64_CALL26 Base::uninlinedFunc(int, int)
// ====== End of function invocation ====== //
 ldr	x19, [sp, #16]
 ldp	x29, x30, [sp], #32
 ret
Base::uninlinedFunc(int, int):
 sub	w0, w1, w2
 ret
    \end{minted}
    \caption{Disassembly of uninlined function invocation (Listing \ref{listing:impact-dispatch-definition} \circledbase{pairedThreeLightGreen}{\scriptsize{2}}).}
    \label{listing:impact-dispatch-uninlined-disassembly}
\end{code}


\vspace{2em}
\begin{code}
    \begin{minted}[fontsize=\scriptsize]{text}
main:
 sub	sp, sp, #0x30
 stp	x29, x30, [sp, #16]
 stp	x20, x19, [sp, #32]
 add	x29, sp, #0x10
 ldr	x0, [x1, #8]
 mov	x19, x1
 mov	x1, xzr
 mov	w2, #0xa                   	// #10
 bl	0 <strtol>
    R_AARCH64_CALL26 strtol
 ldr	x8, [x19, #16]
 mov	x20, x0
 mov	x1, xzr
 mov	w2, #0xa                   	// #10
 mov	x0, x8
 bl	0 <strtol>
    R_AARCH64_CALL26 strtol
 ldr	x8, [x19, #24]
 mov	x19, x0
 mov	x1, xzr
 mov	w2, #0xa                   	// #10
 mov	x0, x8
 bl	0 <strtol>
    R_AARCH64_CALL26 strtol
// ===== Start of function invocation ===== //
 adrp	x8, 0 <main>
    R_AARCH64_ADR_PREL_PG_HI21 vtable for Base+0x10
 add	x11, x8, #0x0
    R_AARCH64_ADD_ABS_LO12_NC vtable for Base+0x10
 cmp	w0, #0x0
 adrp	x8, 0 <main>
    R_AARCH64_ADR_PREL_PG_HI21 vtable for Derived+0x10
 add	x8, x8, #0x0
    R_AARCH64_ADD_ABS_LO12_NC vtable for Derived+0x10
 mov	x9, sp
 add	x10, sp, #0x8
 stp	x8, x11, [sp]
 mov	w1, w20
 csel	x0, x10, x9, gt
 mov	w2, w19
 ldr	x8, [x0]
 ldr	x8, [x8]
 blr	x8
// ====== End of function invocation ====== //
 ldp	x20, x19, [sp, #32]
 ldp	x29, x30, [sp, #16]
 add	sp, sp, #0x30
 ret
Base::virtualFunc(int, int):
 sub	w0, w1, w2
 ret
Derived::virtualFunc(int, int):
 sub	w0, w2, w1
 ret
    \end{minted}
    \caption{Disassembly of polymorphic function invocation (Listing \ref{listing:impact-dispatch-definition} \circledbase{pairedFourDarkGreen}{\scriptsize{3}}).}
    \label{listing:impact-dispatch-polymorphic-disassembly}
\end{code}


\vspace{2em}
\begin{code}
    \begin{minted}[fontsize=\scriptsize]{text}
//// Trace of `Extensibility.time_invoke_method_baseline` :
// == microbenchmarks:152 `time_invoke_method_baseline` ==
// >>> a = 5
154         0   LOAD_CONST           1   (5)
            2   STORE_FAST           1   (a)
// >>> b = 6
155         4   LOAD_CONST           2   (6)
            6   STORE_FAST           2   (b)
// >>> _ = None  # Simulate passing arguments
156         8   LOAD_CONST           3   (None)
            10  STORE_FAST           3   (_)
// >>> Extensibility.EXAMPLE.regularFunction
157         12  LOAD_GLOBAL          0   (Extensibility)
            14  LOAD_ATTR            1   (EXAMPLE)
            16  LOAD_ATTR            2   (regularFunction)
            18  POP_TOP                  ()
// >>> return a - b
158         20  LOAD_FAST            1   (a)
            22  LOAD_FAST            2   (b)
            24  BINARY_SUBTRACT          ()
            26  RETURN_VALUE             ()
// =============================================

// == microbenchmarks:160 `time_invoke_method` ==
// >>> a = 5
162         0   LOAD_CONST           1   (5)
            2   STORE_FAST           1   (a)
// >>> b = 6
163         4   LOAD_CONST           2   (6)
            6   STORE_FAST           2   (b)
// >>> return Extensibility.EXAMPLE.regularFunction(a, b)
164         8   LOAD_GLOBAL          0   (Extensibility)
            10  LOAD_ATTR            1   (EXAMPLE)
            12  LOAD_METHOD          2   (regularFunction)
            14  LOAD_FAST            1   (a)
            16  LOAD_FAST            2   (b)
            18  CALL_METHOD          2   ()

    // === microbenchmarks:16 `regularFunction` ====
    // >>> return a - b
     17         0   LOAD_FAST            1   (a)
                2   LOAD_FAST            2   (b)
                4   BINARY_SUBTRACT          ()
                6   RETURN_VALUE             ()
    // =============================================
            20  RETURN_VALUE             ()
// =============================================
    \end{minted}
    \caption{Bytecode trace of Python method invocation, with the \texttt{CALL\_METHOD} and \texttt{RETURN\_VALUE} opcodes being the target of the measurement by subtracting the elapsed time from that of the baseline inlined implementation.}
    \label{listing:impact-dispatch-python-disassembly}
\end{code}

% \chapter{DynamoRIO call results}
\label{chap:dynamorio-calls}


\begin{code}
    \begin{minted}[fontsize=\scriptsize]{text}
 ../../DynamoRIO-Linux-11.90.20236/bin64/drrun -c ../../DynamoRIO-Linux-11.90.20236/samples/bin64/libcountcalls.so -- ./tools/mlir/unittests/Benchmarks/MLIR_IR_Benchmark --benchmark_filter=SimpleConstantFolding
Client countcalls is running
2025-06-06T15:09:26+00:00
Running ./tools/mlir/unittests/Benchmarks/MLIR_IR_Benchmark
Run on (16 X 2799.81 MHz CPU s)
CPU Caches:
  L1 Data 32 KiB (x8)
  L1 Instruction 32 KiB (x8)
  L2 Unified 512 KiB (x8)
  L3 Unified 16384 KiB (x2)
Load Average: 0.33, 0.19, 0.09
Thread 2350 exited - Instrumentation results:
  saw 182378 direct calls
  saw 10348 indirect calls
  saw 192721 returns

Instrumentation results:
  saw 182378 direct calls
  saw 10348 indirect calls
  saw 192721 returns
    \end{minted}
    \caption{DynamoRIO call tracing for constant folding under micro-benchmarking infrastructure only.}
    \label{listing:dynamorio-calls-workload}
\end{code}


\vspace{2em}
\begin{code}
    \begin{minted}[fontsize=\scriptsize]{text}
$ ../../DynamoRIO-Linux-11.90.20236/bin64/drrun -c ../../DynamoRIO-Linux-11.90.20236/samples/bin64/libcountcalls.so -- ./tools/mlir/unittests/Benchmarks/MLIR_IR_Benchmark --benchmark_filter=SimpleConstantFolding
Client countcalls is running
2025-06-06T15:10:45+00:00
Running ./tools/mlir/unittests/Benchmarks/MLIR_IR_Benchmark
Run on (16 X 2799.81 MHz CPU s)
CPU Caches:
  L1 Data 32 KiB (x8)
  L1 Instruction 32 KiB (x8)
  L2 Unified 512 KiB (x8)
  L3 Unified 16384 KiB (x2)
Load Average: 0.39, 0.22, 0.11
Thread 2366 exited - Instrumentation results:
  saw 107238 direct calls
  saw 6028 indirect calls
  saw 113261 returns

Instrumentation results:
  saw 107238 direct calls
  saw 6028 indirect calls
  saw 113261 returns
    \end{minted}
    \caption{DynamoRIO call tracing for micro-benchmarking infrastructure only.}
    \label{listing:dynamorio-calls-baseline}
\end{code}


% \begin{code}
%     \begin{minted}{haskell}
%         newtype Lasagne = Lasagne Int
%             deriving (Show, Num)

%         -- The stacking operating can be considered integer
%         -- addition of the number of layers
%         instance Semigroup Lasagne where
%             (<>) = (+)

%         -- The identity element is the empty (zero-layer) Lasagne
%         instance Monoid Lasagne where
%             mempty = Lasagne 0

%         -- Stacking 5 and 6 layers gives 11 layers:
%         --
%         -- ghci> Lasagne 5 <> Lasagne 6
%         -- Lasagne 11
%     \end{minted}
%     \caption{A Haskell implementation of the Lasagne monoid, which is not an endofunctor.}
%     \label{}
% \end{code}
