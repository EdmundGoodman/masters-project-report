% Suggested LaTeX style template for Masters project report submitted at the
% Department of Computer Science and Technology
%
% Markus Kuhn, May 2022
% (borrowing elements from an earlier template by Steven Hand)

\usepackage[pdfborder={0 0 0}]{hyperref}  % turns references into hyperlinks
\usepackage[vmargin=20mm,hmargin=25mm]{geometry}  % adjust page margins
\usepackage{graphicx} % allows inclusion of PDF, PNG and JPG images
\usepackage{parskip}  % separate paragraphs with vertical space
                      % instead of indenting their first line
\usepackage{setspace} % for \onehalfspacing
\usepackage{refcount} % for counting pages
\usepackage{upquote}  % for correct quotation marks in verbatim text

\newif\ifsubmission % Boolean flag for distinguishing submitted/final version


% ============= %
% Modifications %
% ============= %

%% Configure fonts
% \usepackage{fontspec}
% Default Serif
% \setmainfont{Linux Libertine O}[Ligatures={Common}]
\usepackage{libertine}  % Linux Libertine
% \usepackage{mathpazo}  % Palatino serif and math
% \usepackage{mathptmx}  % Times New Roman serif and math
% Sans Serif
% \usepackage{biolinum}  % Linux biolinum sans serif
% \usepackage{helvet}  % Arial (/helvetica?) sans serif
\newenvironment{defaultsffamily}{%
  \fontfamily{cmss}\selectfont
}{}
% Code
\usepackage{inconsolata}
% \setmonofont{Fira Code}[Contextuals=Alternate,Scale=MatchLowercase]

%% Extra packages
\usepackage{subcaption}
\usepackage{lastpage}       % get a reference to the last page
\usepackage{booktabs}       % professional-quality tables
\usepackage{amsfonts}       % blackboard math symbols
\usepackage{nicefrac}       % compact symbols for 1/2, etc.
\usepackage[nopatch=footnote]{microtype}      % microtypography
\usepackage{xcolor}         % colors
\usepackage{todonotes}      % todo notes
\usepackage[nottoc]{tocbibind}
\usepackage{changepage}

%% Absolute page numbers
\usepackage{zref-user}
\usepackage{zref-abspage}

%% Glossaries
\usepackage[acronym,shortcuts]{glossaries}
\makeglossaries

%% Caption configuration
\usepackage{caption}
\captionsetup{%
    labelfont=bf,%
    % textfont=it,%
    font=small,%
    skip=0pt,%
    width=0.8\textwidth%,
}
% \captionsetup{textfont=it}
% \renewcommand\fbox{\fcolorbox{white}{white}}
\captionsetup[figure]{skip=5pt}
\captionsetup[table]{skip=5pt}
% \captionsetup[listing]{skip=0pt}

%% Code highlighting
\usepackage{minted}
\setminted{%
    linenos,%
    breaklines,%
    autogobble,%
    % frame=lines,%
    % escapeinside=||,
}
% https://tex.stackexchange.com/a/53540
\newenvironment{code}{\captionsetup{type=figure,name=Listing}}{}

%% Configure urls
\usepackage{url}
% \hypersetup{
% 	colorlinks=true,
% 	linkcolor=black,
% 	urlcolor=black,
% 	citecolor=black
% }

%% Configure bibliography
\usepackage[citestyle=ieee]{biblatex}
\addbibresource{references.bib}
% Avoid overflowing URLs (https://stackoverflow.com/a/43593557)
\setcounter{biburllcpenalty}{7000}
\setcounter{biburlucpenalty}{8000}





% ======================================================================= %
% OpenCompl modifications                                                 %
% - <https://github.com/opencompl/paper-template/blob/main/tex/setup.tex> %
% - <https://github.com/opencompl/paper-template/blob/main/paper.tex>     %
% ======================================================================= %

% %%%%%%%%%%%%%%%%%%%%%%%%%%%%%%%%%%%%%%%%%%%%%%%%%%%%%%%%%%%%%%%%%%%%%%%%%%%%%%%
% % Add minted and support custom lexers
% %%%%%%%%%%%%%%%%%%%%%%%%%%%%%%%%%%%%%%%%%%%%%%%%%%%%%%%%%%%%%%%%%%%%%%%%%%%%%%%
% \usepackage{etoolbox}

% \makeatletter
% \ifcsdef{minted@optlistcl@quote}
% {
% \ifwindows
%   \renewcommand{\minted@optlistcl@quote}[2]{%
%     \ifstrempty{#2}{\detokenize{#1}}{\detokenize{#1="#2"}}}
% \else
%   \renewcommand{\minted@optlistcl@quote}[2]{%
%     \ifstrempty{#2}{\detokenize{#1}}{\detokenize{#1='#2'}}}
% \fi

% % similar to \minted@def@optcl@switch
% \newcommand{\minted@def@optcl@novalue}[2]{%
%   \define@booleankey{minted@opt@g}{#1}%
%     {\minted@addto@optlistcl{\minted@optlistcl@g}{#2}{}%
%      \@namedef{minted@opt@g:#1}{true}}
%     {\@namedef{minted@opt@g:#1}{false}}
%   \define@booleankey{minted@opt@g@i}{#1}%
%     {\minted@addto@optlistcl{\minted@optlistcl@g@i}{#2}{}%
%      \@namedef{minted@opt@g@i:#1}{true}}
%     {\@namedef{minted@opt@g@i:#1}{false}}
%   \define@booleankey{minted@opt@lang}{#1}%
%     {\minted@addto@optlistcl@lang{minted@optlistcl@lang\minted@lang}{#2}{}%
%      \@namedef{minted@opt@lang\minted@lang:#1}{true}}
%     {\@namedef{minted@opt@lang\minted@lang:#1}{false}}
%   \define@booleankey{minted@opt@lang@i}{#1}%
%     {\minted@addto@optlistcl@lang{minted@optlistcl@lang\minted@lang @i}{#2}{}%
%      \@namedef{minted@opt@lang\minted@lang @i:#1}{true}}
%     {\@namedef{minted@opt@lang\minted@lang @i:#1}{false}}
%   \define@booleankey{minted@opt@cmd}{#1}%
%       {\minted@addto@optlistcl{\minted@optlistcl@cmd}{#2}{}%
%         \@namedef{minted@opt@cmd:#1}{true}}
%       {\@namedef{minted@opt@cmd:#1}{false}}
% }
% \minted@def@optcl@novalue{customlexer}{-x}
% }
% {
% }
% \makeatother

%%%%%%%%%%%%%%%%%%%%%%%%%%%%%%%%%%%%%%%%%%%%%%%%%%%%%%%%%%%%%%%%%%%%%%%%%%%%%%%
% Base style and command for \circled to print a colored circle
%%%%%%%%%%%%%%%%%%%%%%%%%%%%%%%%%%%%%%%%%%%%%%%%%%%%%%%%%%%%%%%%%%%%%%%%%%%%%%%
% Width is assured to be the same across all characters using the typewriter font which is monospaced
% Zeroing out the inner separator removes padding between content and node (inner sep).
% Zeroing out the outer separator removes space between the node border and its anchors (e.g., east).
% Minimum size was derived on experimentation and it may need adjustment when changing font style/size.
%
% There is no guarantee for the letter ascenders/descenders to baseline when set to char.base, hence adding \strut.
% which is an invisible vertical rule with the height and depth of the parentheses ( and ).
% It ensures that the line height in a line of text is at least as large as if it contained parentheses.

\usepackage{tikz}
\usetikzlibrary{arrows}
\usetikzlibrary{shapes}

\tikzset{
  circledstyle/.style={
    shape=circle,
    #1,
    font=\tt\small,
    inner sep=0pt,
    outer sep=0pt,
    minimum size=1.2em,
    text=black
  }
}

% define a base tikz node for circled commands accepting a fill colour and the node text as arguments
\DeclareRobustCommand{\circledbase}[3][]{%
    \tikz[baseline=(char.base)]{\node[circledstyle, fill=#2] (char) {#3\strut};}%
}

% % listings don't write "Listing" in autoref without this.
% \providecommand*{\listingautorefname}{Listing}

%%%%%%%%%%%%%%%%%%%%%%%%%%%%%%%%%%%%%%%%%%%%%%%%%%%%%%%%%%%%%%%%%%%%%%%%%%%%%%%
% Add minted aliases
%%%%%%%%%%%%%%%%%%%%%%%%%%%%%%%%%%%%%%%%%%%%%%%%%%%%%%%%%%%%%%%%%%%%%%%%%%%%%%%

\usemintedstyle{colorful}

% % Newer versions of minted require the 'customlexer' argument for custom lexers
% % whereas older versions require the '-x' to be passed via the command line.
% \makeatletter
% \ifcsdef{MintedExecutable}
% {
%   % minted v3
%   \newminted[mlir]{tools/lexers/MLIRLexer.py:MLIRLexerOnlyOps}{mathescape}
%   \newminted[xdsl]{tools/lexers/MLIRLexer.py:MLIRLexer}{mathescape, style=murphy}
% }
% {
%   \ifcsdef{minted@optlistcl@quote}
%   {
%     \newminted[mlir]{tools/lexers/MLIRLexer.py:MLIRLexerOnlyOps}{customlexer, mathescape}
%     \newminted[xdsl]{tools/lexers/MLIRLexer.py:MLIRLexer}{customlexer, mathescape, style=murphy}
%   }
%   {
%     \newminted[mlir]{tools/lexers/MLIRLexer.py:MLIRLexerOnlyOps -x}{mathescape}
%     \newminted[xdsl]{tools/lexers/MLIRLexer.py:MLIRLexer -x}{mathescape, style=murphy}
%   }
% }
% \makeatother
% \newminted[mlir]{tools/lexers/MLIRLexer.py:MLIRLexerOnlyOps}{customlexer, mathescape}
% \newminted[xdsl]{tools/lexers/MLIRLexer.py:MLIRLexer}{customlexer, mathescape, style=murphy}

%%%%%%%%%%%%%%%%%%%%%%%%%%%%%%%%%%%%%%%%%%%%%%%%%%%%%%%%%%%%%%%%%%%%%%%%%%%%%%%
% Add colour scheme
%%%%%%%%%%%%%%%%%%%%%%%%%%%%%%%%%%%%%%%%%%%%%%%%%%%%%%%%%%%%%%%%%%%%%%%%%%%%%%%

% We use the following color scheme
%
% This scheme is both print-friendly and colorblind safe for
% up to four colors (including the red tones makes it not
% colorblind safe any more)
%
% https://colorbrewer2.org/#type=qualitative&scheme=Paired&n=4

\definecolor{pairedNegOneLightGray}{HTML}{cacaca}
\definecolor{pairedNegTwoDarkGray}{HTML}{827b7b}
\definecolor{pairedOneLightBlue}{HTML}{a6cee3}
\definecolor{pairedTwoDarkBlue}{HTML}{1f78b4}
\definecolor{pairedThreeLightGreen}{HTML}{b2df8a}
\definecolor{pairedFourDarkGreen}{HTML}{33a02c}
\definecolor{pairedFiveLightRed}{HTML}{fb9a99}
\definecolor{pairedSixDarkRed}{HTML}{e31a1c}
